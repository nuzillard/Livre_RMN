\chapter{Diagrammes énergétiques}
\label{chap:diagram}

La démarche qui va être présentée dans le chapitre suivant,
intitulé "matrice densité", repose
sur l'acceptation par le lecteur d'un certain nombre de règles
très formelles, mais qui ont comme premier avantage de reproduire
les résultats connus pour les systèmes à un spin et fondés sur
la manipulation du vecteur aimantation macroscopique.
Leur second avantage est d'être applicables "les yeux fermés"
en étant sûr qu'au terme de calculs, parfois un peu lourds, certes,
des résultat reflétant la réalité du comportement de systèmes complexes
pourront être obtenus.
Cette "conception algébrique" de la RMN n'est pas nécessairement la préférée de tous,
et la "conception géométrique" du modèle de Bloch semble être pour certains
(le plus grand nombre ?) plus directement accessible car plus visuelle.

L'extension du modèle de Bloch aux systèmes à plusieurs spins relève 
de la connaissance préalable de leur diagramme énergétique.
Ce chapitre a pour but de présenter ces diagrammes
et d'introduire la notion de couplage scalaire, notion qui est centrale
pour l'étude des structures moléculaires par RMN.
La présentation du modèle de Bloch étendu sera faite ultérieurement,
après la présentation du formalisme de la matrice densité, et afin
de faire le lien entre les deux concepts. 

Les diagrammes énergétiques indiquent quels sont les états d'énergie accessibles
à un spin ou un système de spin plongés dans un champ magnétique $\bzerovecloc$.
L'absorption et l'émission d'énergie sous forme d'une onde électromagnétique
par le système s'effectue à des fréquences caractéristiques qui se déduisent
des valeurs des énergies possibles. 
Ces fréquences sont les mêmes que celles observées pour la précession de l'aimantation
dans le modèle de Bloch et son extension aux systèmes à plus de un spin.
Le diagramme énergétique d'un système de spins est donc utile à connaître,
même pour les inconditionnels de la démarche algébrique.
% Comme précédemment nous nous plaçons dans l'hypothèse des systèmes faiblement couplés
% et des impulsions parfaites.

\section{Systèmes à un spin}

Comme indiqué au début de cet ouvrage, la fréquence de résonance $\nu_I$ d'un noyau $I$,
de rapport gyromagnétique $\gamma_I$ soumis au champ local $\bzerovecloc (I)$ est :
\begin{equation}
\nu_I = \frac{\gi\bzerolocs (I)}{2\pi}
\end{equation}
sachant qu'ici $\nu_I$ représente une fréquence mesurable dans le référentiel
du laboratoire (et non pas dans le référentiel tournant).

La résonance du noyau $I$ est liée à l'existence de 2 niveaux énergétiques
qui avaient été notés $E_{\al}$ et $E_{\be}$ liés aux valeurs du nombre quantique $m_s(I)$
$+1/2$ et $-1/2$, respectivement.
En prenant comme référence (énergie 0)  l'énergie moyenne $(E_{\al} + E_{\be})/2 $,
on obtient $E_{\al} = -1/2.h\nu_I$ et $E_{\be} = +1/2.h\nu_I$, soit
\begin{equation}
E/h = -m_s(I) \nu_I
\end{equation}
ce qui aboutit au diagramme énergétique de la figure \ref{fig:diagrami}.
Les états liés à $m_s(I) = +1/2$ et $-1/2$ seront respectivement notés $\al$ et $\be$.

\begin{figure}[hbt]
\begin{center}
\begin{pspicture}(0,-1.5)(5,2.5)
\psline[linewidth=0.05]{->}(0.5,-1.5)(0.5,1.5)
\psline(0,0)(1,0)
\psline(1,0)(2,1)
\psline(1,0)(2,-1)
\psline[linewidth=1mm](2,1)(3,1)
\psline[linewidth=1mm](2,-1)(3,-1)
\psline{<->}(2.5,-1)(2.5,1)
\rput(0.5,2){$E/h$}
\rput(4.5,2){$m_s$}
\rput(3,0){$\nu_I$}
\rput(3.5,1){$\be$}
\rput(3.5,-1){$\al$}
\rput(4.5,1){$-1/2$}
\rput(4.5,-1){$+1/2$}
\end{pspicture}
\caption{\label{fig:diagrami}
\small Diagramme énergétique d'un spin 1/2 isolé $I$}
\end{center}
\end{figure}

Pour rappel, le chemin qui conduit du diagramme énergétique
vers la précession du vecteur $\aimvec$ hors équilibre à la
fréquence $\nu_I$ s'appuie sur le traitement par la mécanique classique du mouvement
de ce vecteur, traitement dont la légitimité est laissée à
l'appréciation (la crédulité ?) du lecteur qui ne maîtriserait pas
les aspects fondamentaux de la mécanique quantique.

En faisant une entorse aux principes exposés dans le préambule,
le raisonnement suivant présente le lien quantique entre niveaux énergétiques
et précession de Larmor.
Cela doit être considéré comme une digression...

Soient $\qal$ et $\qbe$ sont les fonctions propres de l'hamiltonien
\begin{equation}
H = -2\pi\nu_I \cdot I_z 
\end{equation}
avec respectivement comme valeurs propres $E_{\al}=-h\nu_I/2$ et $E_{\be}=+h\nu_I/2$.
L'état d'\emph{une seule} particule décrit par la fonction d'onde
\begin{equation}
\qpsi_0 = \frac{\qal + \qbe}{\sqrt{2}}
\end{equation}
est un état propre de $I_x$.
C'est-à-dire que si on dispose d'un grand nombre de particules
qui sont toutes à l'instant $t=0$ dans l'état $\qpsi_0$,
et si on mesure $\aimxs = \gamma_I I_x$ sur chaque particule,
alors on est \emph{certain} de trouver à chaque fois $1/2.\gamma_I\hbar$.
L'état $\qpsi_0$ va évoluer
au cours du temps, selon le résultat usuel, issu de l'intégration
de l'équation de Schrödinger dépendante du temps.
\begin{eqnarray}
\qal & \xrightarrow{H,t} & \exp(-iE_{\al}t/\hbar)\cdot \qal\\
\qbe & \xrightarrow{H,t} & \exp(-iE_{\be}t/\hbar)\cdot \qbe\\
\qpsi_0 & \xrightarrow{H,t} & \qpsi(t) \nonumber\\
& & = (\exp(-iE_{\al}t/\hbar) \cdot \qal + \exp(-iE_{\be}t/\hbar) \cdot \qbe)/\sqrt{2}
\end{eqnarray}
La probabilité $p_+(t)$ pour que la mesure de $\aimxs$ donne $+1/2.\gamma_I.\hbar$
est fournie par 
\begin{eqnarray}
p_+(t) & = & |\bpsi_0 \qpsi(t)|^2 \nonumber\\
& = & |\exp(-iE_{\al}t/\hbar) + \exp(-iE_{\be}t/\hbar)|^2/4 \nonumber\\
& = & (1 + \cos((E_{\be} -E_{\al})t/\hbar)/2 \label{eqn:interf}\\
& = & (1 + \cos(2\pi\nu_It))/2
\end{eqnarray}
Sachant que la valeur propre $-1/2.\gamma_I\hbar$ de $\aimxs$ est associée
à la fonction d'onde $(\qal - \qbe)/\sqrt{2}$,
un calcul identique au précédent conduit pour la probabilité 
$p_-(t)$ pour que la mesure de $\aimxs$ donne $-1/2.\gamma_I\hbar$
\begin{equation}
p_-(t)  =  (1 - \cos(2\pi\nu_It))/2
\end{equation}

La valeur moyenne de $\aimx$ au temps $t$, observable sur un grand nombre
de noyaux initialement tous dans l'état $\qpsi_0$ est donc
\begin{eqnarray}
<\aimxs(t)> & = & 1/2.\gamma_I\hbar(p_+(t) - p_-(t)) \nonumber\\
& = & 1/2.\gamma_I\hbar\cos(2\pi\nu_It)
\end{eqnarray}

Suivant la même procédure,
\begin{eqnarray}
<\aimys(t)> & = & -1/2.\gamma_I\hbar\sin(2\pi\nu_It)\\
<\aimzs(t)> & = & 0
\end{eqnarray} 
ce qui correspond une rotation à la fréquence $-\nu_I$ de $<\aimvec(t)>$,
en partant à $t=0$ d'une aimantation alignée sur l'axe $Ox$
du référentiel du laboratoire.

Ce raisonnement est très général et peut s'étendre à l'étude de l'action
du champ $\bunvec$.
Les principes la mécanique quantique se substituent parfaitement
à ceux de la mécanique classique sans introduire l'hypothèse supplémentaire
de la nécessaire validité de cette dernière dans le domaine macroscopique.
Le lecteur attentif aura remarqué que tout se joue dans l'équation
\ref{eqn:interf} où l'évolution sinusoïdale de $\aimxs(t)$ est liée à un
effet d'interférence quantique, tout à fait semblable
dans sa forme à une interférence optique.

Fin de la digression...

En résumé, c'est bien parce qu'il y a deux niveaux d'énergie $E_{\al}$
et $E_{\be}$ qu'il y a une précession de l'aimantation transversale à
la fréquence $(E_{\be} - E_{\al})/h$ et qu'il est donc intéressant
de tracer des diagrammes énergétiques.

\section{Systèmes à deux spins}
\subsection{Sans couplage scalaire}
Si deux systèmes sont réunis sans qu'ils interagissent entre eux,
leur énergie d'interaction avec l'extérieur est la somme
des énergies prises séparément.
On considère ici 2 noyaux $I$ et $S$ caractérisés par leur fréquence
de résonance $\nu_I$ et $\nu_S$.
L'énergie d'interaction $E$ s'exprime donc par
\begin{equation}
E/h = -m_s(I)\nu_I -m_s(S)\nu_S
\end{equation}
L'état quantique du système est défini par les valeurs de $m_s(I)$ et $m_s(S)$.
Sous forme symbolique, on note $\al\al$, $\al\be$, $\be\al$ et $\be\be$
les quatre états possibles de ce système composé de deux spins $1/2$,
le premier signe étant relatif à $I$ et le second à $S$.

Toutes les transitions énergétiques envisageables ne sont cependant pas
réalisables par émission ou absorption d'une onde électromagnétique.
Pour savoir quelles sont les transitions autorisées il faut d'abord
définir le nombre quantique total :
\begin{equation}
m_s = m_s(I) + m_s(S)
\end{equation}
pour définir la règle de sélection :
\begin{equation}
|\Delta m_s| = 1
\end{equation}
qui se justifie à la fois par la loi de conservation du moment cinétique
et l'attribution du spin 1 au photon, la particule
associée au rayonnement électromagnétique.
Pour cette raison, les transitions observables sont aussi appelées
transitions "à $\pm 1$ quanta" ou "à simple quanta".

Il en résulte quatre transitions possibles, pour lesquelles
soit $\Delta m_s(I) = 1$ et $\Delta m_s(S) = 0$,
soit $\Delta m_s(S) = 0$ et $\Delta m_s(S) = 1$.
Les deux premières sont dites "transitions de $I$" et
les secondes "transitions de $S$", comme
indiqué sur la figure \ref{fig:diagramis}.

\begin{figure}[hbt]
\begin{center}
\begin{pspicture}(0,-3.5)(14,4)
\psline[linewidth=0.05]{->}(0.5,-3.5)(0.5,3.5)
\rput(0.5,4){$E/h$}
\psline(0,0)(1,0)
\psline(1,0)(1.5,2)
\psline(1,0)(1.5,-2)
\psline(1.5,2)(2.5,2)
\psline(1.5,-2)(2.5,-2)
\psline{<->}(2,-2)(2,2)
\rput(2.5,0){$\nu_I$}
\psline(2.5,2)(3,3)
\psline(2.5,2)(3,1)
\psline(2.5,-2)(3,-1)
\psline(2.5,-2)(3,-3)
\psline[linewidth=1mm](3,3)(4,3)
\psline[linewidth=1mm](3,1)(4,1)
\psline{<->}(3.5,1.05)(3.5,2.95)
\rput(4,2){$\nu_S$}
\psline[linewidth=1mm](3,-3)(4,-3)
\psline[linewidth=1mm](3,-1)(4,-1)
\psline{<->}(3.5,-1.05)(3.5,-2.95)
\rput(4,-2){$\nu_S$}
\rput(4.5,0){
 \rput(0,3){$\be\be$}
 \rput(0,1){$\be\al$}
 \rput(0,-1){$\al\be$}
 \rput(0,-3){$\al\al$}
}
\rput(5.5,0){
 \rput(0,4){$m_s(I)$}
 \rput(0,3){$-1/2$}
 \rput(0,1){$-1/2$}
 \rput(0,-1){$+1/2$}
 \rput(0,-3){$+1/2$}
}
\rput(7,0){
 \rput(0,4){$m_s(S)$}
 \rput(0,3){$-1/2$}
 \rput(0,1){$+1/2$}
 \rput(0,-1){$-1/2$}
 \rput(0,-3){$+1/2$}
}
\rput(8.5,0){
 \rput(0,4){$m_s$}
 \rput(0,3){$-1$}
 \rput(0,1){$0$}
 \rput(0,-1){$0$}
 \rput(0,-3){$+1$}
}
\rput(11.5,0){
 \psline[linewidth=1mm](-2.5,1)(-1.5,1)
 \rput(-1,1){$\be\al$}
 \psline[linewidth=1mm](-0.5,3)(0.5,3)
 \rput(0,3.5){$\be\be$}
 \psline[linewidth=1mm](0.5,-3)(-0.5,-3)
 \rput(0,-3.5){$\al\al$}
 \psline[linewidth=1mm](2.5,-1)(1.5,-1)
 \rput(1,-1){$\al\be$}
 \psline{<->}(-2,1.05)(0,2.95)
 \rput(-1.5,2.5){$\nu_S$}
 \psline{<->}(-2,0.95)(0,-2.95)
 \rput(-1.5,-1.5){$\nu_I$}
 \psline{<->}(2,-1.05)(0,-2.95)
 \rput(1.5,-2.5){$\nu_S$}
 \psline{<->}(2,-0.95)(0,2.95)
 \rput(1.5,1.5){$\nu_I$}
}
\end{pspicture}
 \caption{\label{fig:diagramis}
 Diagramme énergétique d'un système $IS$ de 2 spins non couplés
 }
\end{center}
\end{figure}

Les fréquences de résonance associées $\nu_I$ et $\nu_S$ sont données par
\begin{eqnarray}
h \nu_I & = & E_{\be\be} - E_{\al\be} = E_{\be\al} - E_{\al\al}\\
h \nu_S & = & E_{\be\be} - E_{\be\al} = E_{\al\be} - E_{\al\al}
\end{eqnarray}

Les noyaux $I$ et $S$ n'ont aucune interaction et le spectre observable
ne fera intervenir que leurs propres fréquences.

Les transitions (une dans chaque sens) entre les états énergétiques 
$\al\al$ et $\be\be$ correspondent à $\Delta m_s = \pm 2$ et sont
appelées transitions "à double quanta" ;
celles entre $\al\be$ et $\be\al$ sont des transitions "à 0 quanta".

\subsection{Avec couplage scalaire}

Deux noyaux $I$ et $S$ de même nature (système $IS$ homonucléaire) ou de nature 
différentes (système $IS$ hétéronucléaire) peuvent interagir entre eux par couplage 
scalaire (appelé ainsi parce qu'il s'exprime sous la forme d'un produit scalaire d'opérateurs...).
Cette interaction a pour médiateurs les orbitales moléculaires et les spins électroniques.
Elle est caractérisée à son intensité $J$, appelée constante de couplage
et mesurée en Hertz.

Le couplage est fort si $J$ est comparable ou plus grand 
que la différence des fréquences de résonance des noyaux $I$ et $S$. 
Cela n'est jamais le cas 
pour les systèmes hétéronucléaires et peu fréquent dans les systèmes homonucléaires, 
surtout à des valeurs élevées du champ magnétique $\bzeros$.
Pour les systèmes faiblement couplés, l'expression de l'énergie
du système des deux spins $I$ et $S$ comporte un terme supplémentaire
lié à l'existence du couplage :
\begin{equation}
E/h = -m_s(I)\nu_I -m_s(S)\nu_S + Jm_s(I)m_s(S)
\end{equation}

Les règles de sélection restant valides (par ce que les couplages sont faibles),
quatre transitions sont observables.
Leurs fréquences sont maintenant toutes différentes et données par :
\begin{eqnarray}
\nu_1 & = & (E_{\be\al} - E_{\al\al})/h = \nu_I - J/2 \\
\nu_2 & = & (E_{\be\be} - E_{\al\be})/h = \nu_I + J/2 \\
\nu_3 & = & (E_{\al\be} - E_{\al\al})/h = \nu_S - J/2 \\
\nu_4 & = & (E_{\be\be} - E_{\be\al})/h = \nu_S + J/2
\end{eqnarray}
Comme indiqué dans la figure \ref{fig:diagramisj}.
Les deux premières fréquences sont liées aux transitions à simple ($\pm 1$) quanta
du noyau $I$ et les deux dernières aux transitions à simple quanta
de $S$.
Ce qui différence les fréquences des deux transitions de $I$ (ou de $S$)
réside simplement dans le fait que l'état du spin $S$ (ou de $I$)
est soit $\al$ soit $\be$.
Dans le premier cas la fréquence de la transition est diminuée de $J/2$,
dans le second cas elle est augmentée de $J/2$.
Une transition d'un noyau est donc pleinement définie à partir de la
donnée de l'état de l'autre noyau du système, ou des autres noyaux s'il
y en a plus que deux dans le système.

\begin{figure}[hbt]
\begin{center}
\begin{pspicture}(0,-3.5)(14,4)
\psline[linewidth=0.05]{->}(0.5,-3.5)(0.5,3.5)
\rput(0.5,4){$E/h$}
\psline(0,0)(1,0)
\psline(1,0)(1.5,2)
\psline(1,0)(1.5,-2)
\psline(1.5,2)(2.5,2)
\psline(1.5,-2)(2.5,-2)
\psline{<->}(2,-2)(2,2)
\rput(2.5,0){$\nu_I$}
\psline(2.5,2)(3,3)
\psline(2.5,2)(3,1)
\psline(2.5,-2)(3,-1)
\psline(2.5,-2)(3,-3)
\psline(3,3)(4,3)
\psline(3,1)(4,1)
\psline{<->}(3.5,1)(3.5,3)
\rput(4,2){$\nu_S$}
\psline(3,-3)(4,-3)
\psline(3,-1)(4,-1)
\psline{<->}(3.5,-1)(3.5,-3)
\rput(4,-2){$\nu_S$}
\psline(4,3)(4.5,3.5)
\psline(4,1)(4.5,0.5)
\psline(4,-1)(4.5,-1.5)
\psline(4,-3)(4.5,-2.5)
\psline[linewidth=1mm](4.5,3.5)(5.5,3.5)
\psline[linewidth=1mm](4.5,0.5)(5.5,0.5)
\psline[linewidth=1mm](4.5,-2.5)(5.5,-2.5)
\psline[linewidth=1mm](4.5,-1.5)(5.5,-1.5)
\rput(5,0){
 \rput(0,3){$\be\be$}
 \rput(0,1){$\be\al$}
 \rput(0,-1){$\al\be$}
 \rput(0,-3){$\al\al$}
}
\rput(6,0){
 \psline{->}(0,3)(0,3.5)
 \psline{->}(0,1)(0,0.5)
 \psline{->}(0,-1)(0,-1.5)
 \psline{->}(0,-3)(0,-2.5)
}
\rput(7,0){
 \rput(0,4){$Jm_s(I)m_s(S)$}
 \rput(0,3.25){$+J/4$}
 \rput(0,0.75){$-J/4$}
 \rput(0,-1.25){$-J/4$}
 \rput(0,-2.75){$+J/4$}
}
\rput(11.5,0){
 \psline[linewidth=1mm](-2.5,0.5)(-1.5,0.5)
 \rput(-1,0.5){$\be\al$}
 \psline[linewidth=1mm](-0.5,3.5)(0.5,3.5)
 \rput(0,4){$\be\be$}
 \psline[linewidth=1mm](0.5,-2.5)(-0.5,-2.5)
 \rput(0,-3){$\al\al$}
 \psline[linewidth=1mm](2.5,-1.5)(1.5,-1.5)
 \rput(1,-1.5){$\al\be$}
 \psline{<->}(-2,0.55)(0,3.45)
 \rput(-2,2){$\nu_S+J/2$}
 \psline{<->}(-2,0.45)(0,-2.45)
 \rput(-2,-1){$\nu_I-J/2$}
 \psline{<->}(2,-1.55)(0,-2.45)
 \rput(2,-2.25){$\nu_S-J/2$}
 \psline{<->}(2,-1.45)(0,3.45)
 \rput(2,1.5){$\nu_I+J/2$}
}
\end{pspicture}
 \caption{\label{fig:diagramisj}
 Diagramme énergétique d'un système $IS$ de 2 spins faiblement couplés.
 }
\end{center}
\end{figure}

La constante de couplage $J$ est expérimentalement accessible en remarquant
que :
\begin{equation}
J = \nu_2 - \nu_1 = \nu_4 - \nu_3 
\end{equation}

Remarquons enfin que les fréquences des transitions à double (DQ) et à zéro (ZQ) quanta
ne dépendent pas de la valeur de $J$ :
\begin{eqnarray}
\nu_{\mbox{\scriptsize{DQ}}} & = & (E_{\be\be} - E_{\al\al})/h = \nu_I + \nu_S \\
\nu_{\mbox{\scriptsize{ZQ}}} & = & (E_{\be\al} - E_{\al\be})/h = \nu_I - \nu_S
\end{eqnarray}

\section{Systèmes à trois spins}
L'énergie d'un système de trois spins faiblement couplés $ISL$ est donnée par :
\begin{eqnarray}
E/h & = & -m_s(I) \nu_I -m_s(S) \nu_S -m_s(L) \nu_L \\
& & \quad + J_{IS} m_s(I) m_s(S) + J_{IL} m_s(I) m_s(L) + J_{SL} m_s(S) m_s(L)
\end{eqnarray}
chacune des constantes de couplage étant faible devant la différence
des fréquences de résonance des noyaux dont elle décrit le couplage.
Un système de trois spins possède donc 8 niveaux énergétiques possibles.
Les transitions autorisées sont celles pour lesquelles
\begin{equation}
|\Delta m_s| = 1
\end{equation}
avec
\begin{equation}
m_s = m_s(I) + m_s(S) + m_s(L)
\end{equation}
où, à nouveau, $m_s$ est le nombre quantique associé à la projection sur
un axe du moment cinétique total du système.

La figure \ref{fig:diagramisl} représente symboliquement les 8 niveaux 
et les douze transitions observables dans un système $ISL$ faiblement couplé.

\begin{figure}[hbt]
\begin{center}
\begin{pspicture}(-4,-3.5)(4,3.5)
\psline[linewidth=1mm](-0.5,3)(0.5,3)
\psline[linewidth=1mm](-3.5,1)(-2.5,1)
\psline[linewidth=1mm](-0.5,1)(0.5,1)
\psline[linewidth=1mm](2.5,1)(3.5,1)
\psline[linewidth=1mm](-0.5,-3)(0.5,-3)
\psline[linewidth=1mm](-3.5,-1)(-2.5,-1)
\psline[linewidth=1mm](-0.5,-1)(0.5,-1)
\psline[linewidth=1mm](2.5,-1)(3.5,-1)
\rput(0,3.5){$\be\be\be$}
\rput(-4,1.5){$\al\be\be$}
\rput(-0.6,1.5){$\be\al\be$}
\rput(4,1.5){$\be\be\al$}
\rput(0,-3.5){$\al\al\al$}
\rput(4,-1.5){$\be\al\al$}
\rput(0.6,-1.5){$\al\be\al$}
\rput(-4,-1.5){$\al\al\be$}
\psline{<->}(0,2.95)(-3,1.05)
\psline{<->}(0,2.95)(0,1.05)
\psline{<->}(0,2.95)(3,1.05)
\psline{<->}(-3,0.95)(-3,-0.95)
\psline{<->}(-3,0.95)(0,-0.95)
\psline{<->}(-3,-0.95)(0,0.95)
\psline{<->}(0,-2.95)(3,-1.05)
\psline{<->}(0,-2.95)(0,-1.05)
\psline{<->}(0,-2.95)(-3,-1.05)
\psline{<->}(3,-0.95)(3,0.95)
\psline{<->}(3,-0.95)(0,0.95)
\psline{<->}(3,0.95)(0,-0.95)
\end{pspicture}
 \caption{\label{fig:diagramisl}
 Diagramme énergétique d'un système de 3 spins $ISL$.
 }
\end{center}
\end{figure}

Ainsi, par exemple, la transition $\be\al\al \rightarrow \be\be\al$ est une transition
du noyau $S$ qui s'opère avec $I$ dans l'état $\be$ et $L$ dans l'état $\al$.
Sa fréquence est la somme de trois termes : $\nu_S$ parce que c'est une transition
de $S$, $+J_{IS}/2$ car $I$ est dans l'état $\be$ et $-J_{SL}/2$ car $L$ est dans
l'état $\alpha$, soit au total 
\begin{equation}
\nu(\be\al\al \rightarrow \be\be\al) = \nu_S + J_{IS}/2 - J_{SL}/2
\end{equation}

La transition $\al\al\be \rightarrow \be\be\al$ satisfait à la condition
$|\Delta m_s| = 1$ bien que sa probabilité d'observation soit nulle.
Il y a au total trois transitions de ce genre qui sont inobservables.
Dans les systèmes fortement couplés $ABC$ il est possible d'observer
jusqu'à 15 (12 + 3) transitions.

\section{Diagramme énergétique et population des états}

Les différents niveaux énergétiques d'un système en équilibre thermodynamique
sont peuplés par les noyaux en fonction de la distribution de Boltzman.
Comme il a déjà été souligné, les différences d'énergie entre niveaux sont
très faibles par rapport $kT$ dans les conditions usuelles de température,
et donc les noyaux se répartissent à peu près équitablement
entre les niveaux possibles. 
Cet "à peu près" fait toute la différence puisque
l'intensité du signal observable est proportionnelle aux différences de population,
comme cela a été étudié en détail pour un spin isolé.

Un système de $n$ spins 1/2 possède $2^n$ niveaux énergétiques puisque chaque spin
peut être soit dans l'état $\al$ soit dans l'état $\be$.
Chaque niveau d'énergie $E_i (1 \le i \le 2^n)$ d'un ensemble de $P$ systèmes
de spins identiques est peuplé par $p_i$ noyaux avec
\begin{eqnarray}
\sum_i^{2^n} p_i & = & P \\
p_i & = & Z \exp(-E_i/kT) = Z(1 - E_i/kT)
\end{eqnarray}
dans l'hypothèse où $E_i \ll kT$.
$Z$ est un facteur de proportionnalité qui est déterminé par
\begin{eqnarray}
P & = & \sum_i^{2^n} (Z - ZE_i/kT) \\
& = & 2^n Z - Z/kT \sum_i^{2^n} E_i \\
& = & 2^n Z
\end{eqnarray}
sachant que la somme des énergies de tous les niveaux est nulle (il
suffit de faire l'addition pour s'en rendre compte).
Deux états indexés $i$ et $j$ d'énergie $E_i$ et $E_j$ auront comme population
$p_i = Z(1-E_i/kT)$ et $p_j = Z(1-E_j/kT)$ soit une différence de population
\begin{equation}
\label{eqn:diffpop}
p_i - p_j = -P/2^n.(E_i - E_j)/kT
\end{equation}
proportionnelle à leur différence d'énergie.
Il est utile de signaler qu'étant donné que les constantes de couplage
scalaire sont au plus de quelques centaines de Hz (sauf cas exceptionnels)
et que les fréquences de résonance sont de l'ordre de quelques dizaines
ou quelques centaines de MHz, les populations ne sont pas affectées par
l'intensité des couplages.

En considérant par exemple le noyau $I$ d'un système $ISL$ ($n$ = 3),
les différences de population $p_{\al i j} -p_{\be i j}$, avec $i$ et $j$
valant $\al$ ou $\be$, sont toutes égales :
\begin{eqnarray}
p_{\al i j} -p_{\be i j} & = & P/8.h\nu_I/kT \\
& = & P/8.\gamma_I.\hbar \bzeros/kT
\end{eqnarray}
sachant que les écarts énergétiques apportés par les déplacements chimiques
($E/h$ est de l'ordre de quelques dizaines de kHz tout au plus)
sont aussi trop faibles pour perturber notablement les populations.
Les différences de population considérées ici sont d'une grande importance pratique
car elles correspondent à des états pour lesquels des transitions sont
observables et car les intensités de ces transitions sont proportionnelles
aux différences de population correspondantes.

Il est possible de définir la différence de population $\Delta P(I)$ du noyau
$I$ comme étant la somme des différences de population correspondant
aux quatre états possibles des spins $S$ et $L$ :
\begin{eqnarray}
\Delta P(I) & = & (p_{\al \al \al} -p_{\be \al \al}) + (p_{\al \al \be} -p_{\be \al \be})
\nonumber\\
& & \quad + (p_{\al \be \al} -p_{\be \be \al}) + (p_{\al \be \be} -p_{\be \be \be}) \\
& = & P.\gamma_I.\hbar B_0/2kT
\end{eqnarray}
Cette valeur est indépendante du nombre de spins dans le système : le facteur 2
au dénominateur provient du rapport $2^n/2^{n-1} = 2$, sachant que $2^n$ provient
de l'équation \ref{eqn:diffpop} et que $2^{n-1}$ est le nombre d'états possibles
une fois que l'état d'un noyau a été fixé.

Le concept de différence de population pour un noyau est utile pour définir
la contribution de chaque noyau $I$ au vecteur aimantation macroscopique
global de l'échantillon $\aimzeros$ dans sa situation d'équilibre.
Ainsi,
\begin{eqnarray}
\aimzerozs(I) & = & \Delta P(I) . \gamma(I)\hbar/2 \\
& = & P.\gamma(I)^2.\hbar^2 B_0/4kT
\end{eqnarray}
le résultat établi pour un spin 1/2 isolé restant valable pour un spin 1/2
appartenant à un système de spins couplés.
Il faut toutefois noter que dans le cas d'un système à $n$ spins, 
la contribution de chaque noyau à $\aimzerovec$
résulte de la sous-contribution de $2^{n-1}$ vecteurs aimantation macroscopique
identiques (vecteurs élémentaires), 
chacun étant associé à une énergie d'une transition observable,
c'est-à-dire à une fréquence de résonance mesurable.

La mise hors équilibre de l'aimantation de l'échantillon par création
d'aimantation transversale est le moyen de mettre en évidence
les transitions entre niveaux énergétiques.
Chaque vecteur élémentaire décrit ensuite le mouvement de précession de Larmor
à la fréquence égale à la fréquence de la transition qui lui correspond.
Tout ceci constitue la base du modèle vectoriel de la RMN, généralisé aux systèmes
faiblement couplés. Le modèle vectoriel sera étudié plus en détail, après avoir
exposé la description de la RMN à l'aide du concept de matrice densité.
