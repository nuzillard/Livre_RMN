\chapter{Préambule}

De toute évidence, le document que vous lisez actuellement ressemble à un livre
qui parle de Résonance Magnétique Nucléaire.
\emph{C'est avant tout un ouvrage en cours d'écriture.}
Il a pour origine ma propre découverte à la fois de la RMN et d'un outil théorique
qui m'a permis de faire un lien entre les incantations rituelles 
effectuées autour d'un spectromètre et le résultat qui en sort, 
à savoir un spectre qui renseigne sur la structure d'une substance encore inconnue.

Les premières lignes de cet ouvrage ont été écrites vers la fin des années 1980
sur un MacIntosh avec un logiciel de traitement de texte très répandu alors
(et encore aujourd'hui).
Le texte initial a dormi pendant des années. 
J'en ai changé de temps en temps le support
physique (une disquette) pour le préserver des outrages du temps.
Dans sa forme actuelle, la composition typographique est réalisée avec \LaTeX,
dans un environnement informatique libre, Linux. 
Les quelques figures présentes sont réalisées 
en Postscript (écrit à la main ou produit par programme), 
ou avec \LaTeX directement et son étonnant package PSTricks.

Le texte qui suit est certainement très incomplet, il contient des fautes
de typographie, d'orthographe et de syntaxe ; il y manque des figures qui seraient
bien venues pour illustrer certains propos un peu arides.
C'est un ouvrage en évolution lente, que j'enrichis quand j'en ai le loisir.

Le but recherché est de présenter la RMN à un niveau élémentaire (le mien),
en ayant pour bagage quelques connaissances de physique de base supportées par 
un peu de mathématiques, surtout dans les chapitres \ref{chap:intro} et \ref{chap:bloch}, 
à l'occasion de la transformation de Fourier et du traitement du signal.
Une présentation des diagrammes énergétiques est donnée au chapitre \ref{chap:diagram}.
Le volumineux chapitre \ref{chap:densite} présente les règles de calcul qui permettent d'analyser
la majorité des séquences impulsionnelles courantes en RMN des liquides, 
tant que la relaxation n'y intervient pas.
Il présente aussi les bases du cyclage des phases et l'utilisation des
impulsions de gradient de champs statique.
Les chapitres suivants analysent le fonctionnement des expériences les plus
courantes en RMN des liquides.
Un chapitre sur l'attribution de spectres et l'élucidation structurale est prévu,
mais il ne fait pas partie du texte initial, comme d'ailleurs l'utilisation
des impulsions de gradient de champ pour l'imagerie.
Les chapitres qui sont déjà écrits ne sont pas pour autant définitivement fixés
dans le marbre et des ajouts seront encore nécessaires.

Conscient de l'effort soutenu que nécessite l'écriture d'un \emph{vrai}
livre et du caractère aléatoire de sa publication
dans un contexte où de nombreux ouvrages sont disponibles (y compris en français),
j'ai décidé, comme d'autres, de laisser mon texte
à la disposition de l'internaute spectroscopiste qui voudra bien en prendre
connaissance, voire tenter d'y apprendre quelque chose.

Bonne lecture !\\[1ex]

Jean-Marc Nuzillard
