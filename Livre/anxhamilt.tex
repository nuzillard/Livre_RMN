\chapter{Évolution des états et relations de commutation}
\label{chap:hamilt}

\section{Principe}
Cette annexe est destinée à exposer le lien entre les règles de calcul
de l'évolution de la matrice densité d'un système et les
relations de commutation de l'hamiltonien d'évolution
(indépendant du temps) avec l'opérateur qui caractérise l'état initial du système.
Cela donne l'opportunité de sortir du cadre particulier
où les termes d'un hamiltonien composé de plusieurs termes doivent
impérativement tous commuter entre eux.
Les applications possibles vont de l'étude des systèmes homonucléaires
fortement couplés aux effets d'offset et au mélange isotrope.
Nous allons donc considérer le problème de l'évolution
d'un état $A$ sous l'action d'un hamiltonien $H$ appliqué
pendant un temps $t$.

Dans cette annexe, les règles de commutation des opérateurs sont écrites à l'aide de
la notation "officielle", à savoir $[A,B]$ pour le commutateur de $A$ et de $B$,
sachant que
\begin{equation}
[A,B] = i\{A,B\} = AB - BA
\end{equation}
Cette définition, même si elle reste abstraite tant que les matrices des opérateurs
ne sont pas connues de manière explicite, indique trois propriétés de base des commutateurs :
\begin{equation}
[A,B+C] = [A,B] + [A,C] \quad [\lambda A,B] = \lambda[A,B] \quad [A,B] = -[B,A]
\end{equation}

\subsection{Rappel des axiomes}
Les relations qui servent à calculer l'évolution de la matrice de base
d'un système 
\emph{dans la base des produits d'opérateurs cartésiens}
ont été introduites au paragraphe \ref{subsec:transf} :
\begin{eqnarray}
\label{eqn:rien}
\mbox{si}\quad [H,A] = 0 & \qalorsq & 
A \stackrel{Ht}{\Longrightarrow} A \\
\label{eqn:evolcart}
\mbox{si}\quad [H,A] \ne 0 & \qalorsq & 
A \flham{Ht} \cos(at)A + \sin(at)B
\end{eqnarray}
où $a$ est un nombre réel qui caractérise $H$.
Ces relations, que le lecteur a été obligé d'admettre, constituent de fait
des axiomes de base pour les calculs d'évolution de l'état des
systèmes de spins.
Elles se justifient bien entendu à partir des principes de
base de la mécanique quantique et de la définition de l'opérateur densité.
D'autres formulations de ce qui est ici considéré comme des
axiomes sont possibles et cette annexe
a pour but de les énoncer.

L'équation \ref{eqn:rien} n'appelle aucun commentaire particulier
(la double flèche souligne l'invariance de $A$).
L'équation \ref{eqn:evolcart} devient générale quand elle est écrite :

\noindent
\begin{minipage}{0.5cm}si\end{minipage}
\hfill
\parbox{3.5cm}
{\begin{eqnarray*} [H,A] & = & iaB \\ {[H,B]} & = & -iaA \end{eqnarray*}}
\hfill\hfill
\begin{minipage}{0.5cm}alors\end{minipage}
\hfill
\parbox{8cm}{
\begin{eqnarray*}
A & \flham{Ht} & \cos(at)A + \sin(at)B \\
B & \flham{Ht} & \cos(at)B - \sin(at)A
\end{eqnarray*}}
\parbox{1cm}{\begin{eqnarray}\label{eqn:ordre2}\end{eqnarray}}
\\
\noindent
ce qui correspond effectivement à la situation rencontrée avec
les produits d'opérateurs cartésiens lorsque $H$ ne comprend
qu'un seul terme.
L'axiome \ref{eqn:ordre2} définit l'action du superopérateur
relatif à l'application de $H$ pendant le temps $t$ sur les opérateurs
$A$ et $B$.
Toute combinaison de ces opérateurs évolue en donnant une combinaison de ces
mêmes opérateurs.
Mathématiquement, le sous-espace engendré par $A$ et $B$ est stable
par action du superopérateur d'évolution.
Puisque ce sous-espace est de dimension 2, l'axiome \ref{eqn:ordre2}
définit ce qu'il convient d'appeler un problème d'ordre 2.

\ignore{
Lorsque $H$ est composé d'une somme de termes
\emph{qui commutent tous entre eux par paires}
il suffit d'appliquer à l'état initial les superopérateurs
relatifs aux différents termes un à un dans un ordre quelconque.
}
Une autre formulation de l'axiome \ref{eqn:ordre2} est donnée par :

\noindent
\begin{minipage}{0.5cm}si\end{minipage}
\hfill
\parbox{3.5cm}
{\begin{eqnarray*} [H,A] & = & -aB \\ {[H,B]} & = & -aA \end{eqnarray*}}
\hfill\hfill
\begin{minipage}{0.5cm}alors\end{minipage}
\hfill
\parbox{8cm}{
\begin{eqnarray*}
A & \flham{Ht} & \cos(at)A + i\sin(at)B \\
B & \flham{Ht} & \cos(at)B + i\sin(at)A
\end{eqnarray*}}
\parbox{1cm}{\begin{eqnarray}\label{eqn:ordre2b}\end{eqnarray}}
\\
\noindent
qui s'obtient à partir de \ref{eqn:ordre2} en remplaçant $B$ par $iB$
et $a$ par $-a$.
Ainsi :
\begin{eqnarray}
[\pi J 2I_zS_z, I_-] & = & [\pi J 2I_zS_z, I_x - iI_y] \nonumber\\
& = & i\pi J 2I_yS_z - \pi J 2I_xS_z \nonumber\\
& = & -\pi J 2I_-S_z
\end{eqnarray}
De la même manière,
\begin{equation}
[\pi J 2I_zS_z, 2I_-S_z] = -\pi J I_-
\end{equation}
d'où le résultat établi précédemment 
(équations \ref{eqn:evolipm-is} et \ref{eqn:evolipmsz-is}) :
\begin{eqnarray}
\label{eqn:evolim-is}
I_- & \flham{\pi J t 2I_zS_z} &
\cos(\pi J t)I_- + i \sin(\pi J t) 2I_-S_z \\
2I_-S_z & \flham{\pi J t 2I_zS_z} &
\cos(\pi J t)2I_-S_z + i \sin(\pi J t) I_-
\end{eqnarray}

\subsection{Axiomes équivalents}
En effectuant la somme membre à membre des commutateurs de
l'axiome \ref{eqn:ordre2b} on obtient
\begin{equation}
[H,A+B] = -a(A+B)
\end{equation}
avec pour conséquence :
\begin{equation}
A+B \flham{Ht} \exp(iat)(A+B)
\end{equation}
pour tout état initial $A+B$.

L'axiome \ref{eqn:ordre1} :
\begin{equation}
\label{eqn:ordre1}
\mbox{si}\quad [H,A]=-aA \qalorsq
A \flham{Ht} \exp(iat)A
\end{equation}
est équivalent à \ref{eqn:ordre2} tout en ne faisant
intervenir qu'un seul opérateur $A$.

En effet, si
\begin{equation}
[H,A] = iaB \qetq [H,B] = -iaA
\end{equation}
(hypothèse de \ref{eqn:ordre2}) alors
\begin{equation}
[H,A] = a(iB) \qetq [H,iB] = aA
\end{equation}
d'où, en considérant la somme et la différence membre à membre :
\begin{eqnarray}
[H,A+iB] & = & a(A+iB) \\
{[H,A-iB]} & = & -a(A-iB)
\end{eqnarray}
avec pour conséquence
\begin{eqnarray}
A+iB & \flham{Ht} & \exp(-iat)(A+iB) \\
A-iB & \flham{Ht} & \exp(iat)(A-iB)
\end{eqnarray}
soit encore, en calculant les demi sommes et différences
\begin{eqnarray}
A & \flham{Ht} & \cos(at)A + \sin(at)B \\
B & \flham{Ht} & \cos(at)B - \sin(at)B
\end{eqnarray}
qui est la solution attendue à un problème d'ordre 2.

L'axiome \ref{eqn:ordre1} définit un problème d'ordre 1, vu
que tout multiple de $A$ est transformé en un multiple
de l'unique opérateur $A$.
Il est possible de démontrer que tout calcul d'évolution
d'un état peut se ramener à un ensemble de problèmes d'ordre 1.
Les relations \ref{eqn:ordre2} et \ref{eqn:ordre2b} en sont des conséquences
pratiques à utiliser.

La relation \ref{eqn:ordre1} s'applique lorsque
lorsque $A$ est une cohérence du système étudié, subissant
l'effet de l'hamiltonien d'évolution libre, en vertu
de la définition même d'une cohérence.
Ainsi, le résultat
\begin{equation}
I_- \flham{\omsi t I_z} \exp(i\Omega t)I_-
\end{equation}
est une conséquence immédiate de la relation de commutation
\begin{equation}
[\omsi I_z, I_-] = -\omsi I_-
\end{equation}

\subsection{Un problème général d'ordre 2}
Le problème considéré dans ce paragraphe correspond aux relations de commutation
\begin{equation}
\label{eqn:ordre2chyp}
[H,A] = -aA -bB \qetq [H,B] = -bA + aB
\end{equation}

Il existe une méthode un peu longue mais systématique pour prédire
comment vont évoluer $A$ et $B$.
En remarquant que dans les cas précédents $[H,[H,A]]$ était un multiple de $A$,
le calcul de cette quantité donne ici :
\begin{eqnarray}
[H,-aA-bB] & = & -a(-aA-bB) -b(-bA+aB) \\
& = & (a^2+b^2)A \\
& = & c^2 A
\end{eqnarray}
en posant
\begin{equation}
c = \sqrt{a^2+b^2}
\end{equation}

Les hypothèses \ref{eqn:ordre2chyp} peuvent alors être réécrites
\begin{equation}
[H,A] = -c\frac{aA+bB}{c} \qetq [H,\frac{aA+bB}{c}] = -cA
\end{equation}
ce qui d'après \ref{eqn:ordre2b} implique :
\begin{equation}
A \flham{Ht} \cos(ct)A + i\sin(ct)\frac{aA+bB}{c}
\end{equation}
soit
\begin{equation}
\label{eqn:ordre2ca}
A \flham{Ht} 
\left(\cos(ct)+i\frac{a}{c}\sin(ct)\right)A + i\frac{b}{c}\sin(ct)B
\end{equation}
En remarquant que dans \ref{eqn:ordre2chyp} $A$ est permuté avec $B$
lorsque $a$ est changé en $-a$ :
\begin{equation}
\label{eqn:ordre2cb}
B \flham{Ht} 
\left(\cos(ct)-i\frac{a}{c}\sin(ct)\right)B + i\frac{b}{c}\sin(ct)A
\end{equation}

\section{Applications}

\subsection{Couplages forts}
Le calcul des spectres $AB$ où deux noyaux $A$ et $B$ forment un système
de spins fortement couplés est un grand "classique" de l'application du formalisme de la
mécanique quantique à la RMN.
L'approche développée dans cette annexe nous permet d'obtenir le même
résultat, sans introduire ni diagonalisation d'opérateur hamiltonien
ni calcul de moment de transition.
Nous continueront d'appeler $I$ et $S$ les noyaux couplés, $\nu_I$ et $\nu_S$
leur fréquence de résonance et $J$ leur constante de couplage.

L'opérateur hamiltonien complet de ce système est :
\begin{eqnarray}
H & = & 2\pi\nu_I I_z + 2\pi\nu_S S_z + 
2\pi J \boldsymbol{\vec{I}} \cdot \boldsymbol{\vec{S}} \\
\label{eqn:hcouplefort}
& = & 2\pi\nu_I I_z + 2\pi\nu_S S_z + 
\pi J (2I_xS_x + 2I_yS_y + 2I_zS_z)
\end{eqnarray}
écriture qui fait apparaître le couplage (scalaire) comme résultant du 
produit scalaire des opérateurs vectoriels liés à la mesure du moment cinétique
de spin de $I$ et de $S$.

Si le but recherché n'est que de décrire l'évolution de l'aimantation du système à
l'issue d'une impulsion RF et pendant l'acquisition du signal,
il suffit de prévoir comment évolue $I_- + S_-$ sous l'action de $H$
pendant le temps $t$.
Toutefois les équations qui seront écrites permettent d'aller plus loin,
notamment de décrire l'effet des couplages forts dans certaines
expériences de RMN 2D.

Dans un système $IS$, les deux noyaux ont des fréquences de résonance
différentes l'une de l'autre, même si l'expression 
\ref{eqn:hcouplefort} de $H$ traduit l'existence
d'une certaine symétrie.
L'écriture
\begin{equation}
\label{eqn:hcouplefort2}
H = \pi(\nu_I + \nu_S)(I_z + S_z) + \pi J \, 2I_zS_z + \pi(\nu_I - \nu_S)(I_z - S_z)
+ \pi J (2I_xS_x + 2I_yS_y)
\end{equation}
fait apparaître quatre termes.
Les deux premiers sont totalement symétriques
vis-à-vis de $I$ et $S$, le troisième concentre toute la non-symétrie de $H$,
alors que le quatrième est celui qui est négligé lorsque le couplage est faible,
c'est-à-dire lorsque $|J| \ll |\nu_I - \nu_S|$.
Il suffit de supprimer le quatrième terme dans \ref{eqn:hcouplefort2} pour
retrouver l'expression \ref{eqn:hcouplefaible}.
Le choix de la décomposition \ref{eqn:hcouplefort2} de $H$ n'est pas évident \emph{a priori}
et il ne faut pas que le lecteur se vexe de ne pas l'avoir trouvée lui-même.
 
Les notations
\begin{eqnarray}
H & = & H_0 + H_1 + H_2 \\
H_1 & = & \pi J 2I_zS_z \\
H_2 & = & a H_3 + b H_4 \qavecq a = \pi J 
\qetq b = \pi(\nu_I - \nu_S) \\
H_3 & = & 2I_xS_x + 2I_yS_y\\
H_4 & = & I_z - S_z
\end{eqnarray}
simplifient les expressions qui vont suivre.
Connaissant l'évolution de $I_-$, il suffit de changer $b$ en $-b$
pour trouver celle de $S_-$.
Il est facile de vérifier que
\begin{equation}
[H_0, H_1] = [H_0, H_2] = [H_1, H_2] =  0 
\end{equation}
à partir des relations de commutation entre produits d'opérateurs
cartésiens.
Il faut toutefois ajouter à la liste celles qui sont connues, celles
du type $[I_z, 2I_xS_x] = i \, 2I_yS_x$ qui obéissent à la même
logique que $[2I_zS_z, I_x] = i \, 2I_yS_z$.

La nullité des commutateurs de $H_0$, $H_1$ et $H_2$ pris deux par deux
permet d'appliquer successivement ces trois opérateurs pendant le temps $t$.
Nous cherchons donc $\sigma_3$ tel que :
\begin{equation}
\sigma_0 = I_-\flham{H_0 t} \sigma_1
\flham{H_1 t} \sigma_2
\flham{H_2 t} \sigma_3
\end{equation}
Sachant que $[I_z + S_z, I_-] = -I_-$,
\begin{equation}
\sigma_1 = \exp(i\pi(\nu_I+\nu_S)t) \sigma_0
\end{equation}
il suffit de calculer $\sigma_5$ :
\begin{equation}
\sigma_0 = I_- \flham{H_1 t} \sigma_4
\flham{H_2 t} \sigma_5
\end{equation}
et d'écrire :
\begin{equation}
\label{eqn:couplefortsigma3}
\sigma_3 = \exp(i\pi(\nu_I+\nu_S)t) \sigma_5
\end{equation}

L'opérateur $H_2$ est constitué de deux termes qui ne commutent
pas entre eux.
A partir des relations :

\noindent\hfill
\parbox{6cm}{
\[
\begin{array}{r@{\,=\,}l@{\quad\quad}r@{\,=\,}l}
[H_3, I_-] & 2I_zS_- & [H_4, I_-] & -I_- \\
{[H_3, 2I_zS_-]} & I_- & [H_4, 2I_zS_-] & 2I_zS_-
\end{array}
\]
}
\hfill
\parbox{1cm}{
\begin{eqnarray}\end{eqnarray}
}

\noindent
on déduit les relations de commutation pour $H_2$ :
\begin{eqnarray}
[H_2, I_-] & = & -b \, I_- + a \, 2I_zS_- \\
{[H_2, 2I_zS_-]} & = & a \, I_- + b \, 2I_zS_-
\end{eqnarray}
qui se ramènent aux hypothèses \ref{eqn:ordre2chyp} moyennant les transformations :
\begin{equation}
I_- \rightarrow A \quad 2I_-Sz \rightarrow B \quad
-b \rightarrow -a \quad a \rightarrow -b
\end{equation}
avec pour conséquence :
\begin{equation}
\label{eqn:evolimh2}
I_- \flham{H_2 t}
\left(\cos ct + i\frac{b}{c}\sin ct \right) \, I_- -i\frac{a}{c}\sin ct \, 2I_zS_-
\end{equation}

De même, les relations de commutation :
\begin{eqnarray}
[H_2, 2I_-S_z] & = & -b \, I_-S_z + a \, S_- \\
{[H_2, S_-]} & = & a \, I_-S_z + b \, S_-
\end{eqnarray}
conduisent à :
\begin{equation}
\label{eqn:evolimszh2}
2I_-S_z \flham{H_2 t}
\left(\cos ct + i\frac{b}{c}\sin ct \right) \, 2I_-S_z -i\frac{a}{c}\sin ct \, S_-
\end{equation}

L'hamiltonien $H_2$ transforme donc de l'aimantation transversale du noyau $I$
en aimantation transversale de $S$ et réciproquement.
Ceci n'a pas lieu lorsque le système est faiblement couplé
et est en fait dû aux propriétés de commutation de $H_3$.
D'après \ref{eqn:evolim-is} et dans le contexte de ce paragraphe,
\begin{equation}
\sigma_4 = \cos at \, I_- + i\sin at \, 2I_-S_z
\end{equation}
Les équations \ref{eqn:evolimh2}, \ref{eqn:evolimszh2} 
puis \ref{eqn:couplefortsigma3} fournissent le moyen
de calculer $\sigma_5$ puis $\sigma_3$ à partir de $\sigma_4$ :
\begin{eqnarray}
\lefteqn{\sigma_3 = \exp(i \pi (\nu_I + \nu_S)t) \Bigg[ (\cos at\left(\cos ct + 
i\frac{b}{c}\sin ct\right) \, I_- +
\frac{a}{c}\sin at \sin ct \, S_-} \nonumber\\
& & + \sin at\left(\cos ct + i\frac{b}{c}\sin ct\right) \, 2 I_-S_z
- i\frac{a}{c}\cos at \sin ct \, 2I_zS_- \Bigg]
\end{eqnarray}

L'évolution de $I_-$ sous l'action de $H$ préserve bien l'ordre de cohérence total (-1)
bien qu'un état d'ordre de cohérence partiel -1 du noyau $I$ ($I_-$) évolue pour donner
un terme d'ordre de cohérence partiel 0 ($2I_zS_-$) de ce même noyau.

A partir de $\sigma_3$, l'expression
du signal $s_I(t)$ qui provient de l'évolution de $I_-$ sous l'action de $H$
pendant le temps $t$ est la somme des des coefficients multiplicatifs
de $I_-$ et $S_-$ dans l'expression de $\sigma_3$ :
\begin{equation}
s_I(t) = \exp(i \pi (\nu_I + \nu_S)t) \Bigg[
\cos at\left(\cos ct + i\frac{b}{c}\sin ct\right) + 
\frac{a}{c}\sin at \sin ct
\Bigg]
\end{equation}

\begin{figure}[hbt]
\begin{center}
\begin{pspicture}(-4.5,-2.5)(10,3.5)
\SpecialCoor
\psline{<->}(-3.6,3)(-1.6,3)
\uput[90](-2.6,3){$J$}
\psline{<->}(3.6,3)(1.6,3)
\uput[90](2.6,3){$J$}
\psline(-2.4,0.5)(-2.4,-1)
\psline(2.4,0.5)(2.4,-1)
\psline{<->}(-2.4,-0.8)(2.4,-0.8)
\uput[90](0,-0.8){$\Delta \nu$}
\psline(-2.6,0.5)(-2.6,-2.2)
\psline(2.6,0.5)(2.6,-2.2)
\psline{<->}(-2.6,-2)(2.6,-2)
\uput{3pt}[90](0,-2){$\displaystyle
\sqrt{\displaystyle J^{\scriptstyle 2} + (\Delta\nu)^{\scriptstyle 2}} $}
\psline(0,-0.1)(0,1.5)
\uput{3pt}[90](0,1.5){$\displaystyle \frac{\displaystyle \nu_I + \nu_S}{\displaystyle 2}$}
\psline[linestyle=dashed,dash=5pt 5pt](-4.5,1.23)(5,1.23)
\uput[0](5,1.23){$\displaystyle I_{\mbox{rel}} = \frac{\displaystyle 1}{\displaystyle 2}
\left(1 - \frac{\displaystyle J}{\displaystyle 
\sqrt{\displaystyle J^{\scriptstyle 2} + (\Delta\nu)^{\scriptstyle 2}}} \right) $}
\psline[linestyle=dashed,dash=5pt 5pt](-4.5,2.77)(5,2.77)
\uput[0](5,2.77){$\displaystyle I_{\mbox{rel}} = \frac{\displaystyle 1}{\displaystyle 2}
\left(1 + \frac{\displaystyle J}{\displaystyle
\sqrt{\displaystyle J^{\scriptstyle 2} + (\Delta\nu)^{\scriptstyle 2}}} \right) $}
\psset{linewidth=1.2pt}
\psline(-4.5,0)(-3.7,0)(-3.6,1.23)(-3.5,0)(-1.7,0)(-1.6,2.77)(-1.5,0)(1.5,0)(1.6,2.77)(1.7,0)(3.5,0)(3.6,1.23)(3.7,0)(4.5,0)
\end{pspicture}
\caption{\label{fig:couplefort}
\small Spectre issu d'un système $IS$ fortement couplé.}
\end{center}
\end{figure}

Le signal $s_S(t)$ qui provient de $S$ se déduit de $s_I(t)$ en remplaçant $b$
par $-b$. En conséquence, le signal détecté $s(t)$ s'écrit :
\begin{eqnarray}
s(t) & = & s_I(t) + s_S(t) \\
& = & 2 \exp(i \pi (\nu_I + \nu_S)t) \left(
\cos at \cos ct + \frac{a}{c} \sin at \sin ct \right)
\end{eqnarray}

Le signal $s(t)$ se décompose comme la somme de deux termes $s_s(t)$ et $s_a(t)$, où
les indices $_a$ et $_s$ signifient \emph{s}ymétrique et \emph{a}nti-symétrique :
\begin{eqnarray}
\lefteqn{s_s(t) = \frac{1}{2} \exp(i \pi (\nu_I + \nu_S)t)} \nonumber\\
& & \quad\quad (\exp(ict)+\exp(-ict))(\exp(iat)+\exp(-iat)) \\
\lefteqn{s_a(t) = -\frac{1}{2}\frac{a}{c} \exp(i \pi (\nu_I + \nu_S)t)} \nonumber\\
& & \quad\quad (\exp(ict)-\exp(-ict))(\exp(iat)-\exp(-iat))
\end{eqnarray}

En se rappelant que
\begin{equation}
a = \pi J \quad\mbox{et}\quad \frac{c}{\pi}=\sqrt{J^2+(\Delta \nu)^2}
\end{equation}
la TF de $s_s(t)$ fournit quatre raies d'intensités
relatives toutes égales à $\frac{1}{2}$ et celle de $s_a(t)$ fournit
quatre raies aux mêmes fréquences mais avec des intensités $\pm\frac{\pi J}{2c}$.
Le spectre issus de la TF de $s(t)$ présente donc les caractéristiques indiquées
dans le tableau \ref{tab:couplefort} et illustrées par la figure
\ref{fig:couplefort}.
La somme des intensités relatives 
de deux raies distantes de $J$ en fréquence
est toujours égale à 1.
Le couplage fort n'affecte donc pas le caractère quantitatif des spectres.

\begin{table}[hbt]
\caption{Spectre d'un système $IS$ fortement couplé}
\label{tab:couplefort}
\begin{center}
\begin{tabular}{cc}
\hline
Fréquence & Intensité \\ \hline
$\frac{1}{2}\left(\nu_I + \nu_S + \frac{c}{\pi} + J \right)$ & 
$\frac{1}{2}\left(1 - \frac{\pi J}{c} \right)$ \\
$\frac{1}{2}\left(\nu_I + \nu_S + \frac{c}{\pi} - J \right)$ & 
$\frac{1}{2}\left(1 + \frac{\pi J}{c} \right)$ \\
$\frac{1}{2}\left(\nu_I + \nu_S - \frac{c}{\pi} + J \right)$ & 
$\frac{1}{2}\left(1 + \frac{\pi J}{c} \right)$ \\
$\frac{1}{2}\left(\nu_I + \nu_S - \frac{c}{\pi} - J \right)$ & 
$\frac{1}{2}\left(1 - \frac{\pi J}{c} \right)$ \\
\hline
\end{tabular}
\end{center}
\end{table}

Deux cas particuliers sont intéressants à mentionner : celui où $J \ll \Delta\nu$
et celui où $\Delta\nu = 0$.
Si $J \ll \Delta\nu$, en faisant une approximation au premier ordre,
$c = \pi \, \Delta\nu$ et alors les fréquences sont identiques à celles d'un système
faiblement couplé et les raies "externes" présentent une intensité relative
un peu plus faible que les raies internes, comme indiqué dans le
tableau \ref{tab:pastropfort}.
Le lecteur pourra vérifier que l'allure du spectre n'est pas modifiée si
$J$ est changé en $-J$.

\begin{table}[hbt]
\caption{Spectre d'un système $IS$ pas trop fortement couplé}
\label{tab:pastropfort}
\begin{center}
\begin{tabular}{cc}
\hline
Fréquence & Intensité \\ \hline
$\nu_I + \frac{J}{2}$ & 
$\frac{1}{2}\left(1 - \frac{J}{\Delta\nu} \right)$ \\
$\nu_I - \frac{J}{2}$ & 
$\frac{1}{2}\left(1 + \frac{J}{\Delta\nu} \right)$ \\
$\nu_S + \frac{J}{2}$ & 
$\frac{1}{2}\left(1 + \frac{J}{\Delta\nu} \right)$ \\
$\nu_S - \frac{J}{2}$ & 
$\frac{1}{2}\left(1 - \frac{J}{\Delta\nu} \right)$ \\
\hline
\end{tabular}
\end{center}
\end{table}

Si $\Delta\nu = 0$ alors les deux noyaux ont la même fréquence de résonance $\nu_{IS}$
et $c = \pi J$. Les raies externes sont d'intensités nulles et les deux raies internes
sont superposées à la fréquence $\nu_{IS}$ avec chacune une intensité relative de 1,
soit 2 au total.
Ce résultat est général : tout se passe comme si la constante de
couplage de deux noyaux magnétiquement équivalents est nulle.

\subsection{Mélange isotrope}
Un système homonucléaire couplé (fortement ou faiblement) 
$IS$ soumis à une suite ininterrompue d'échos de
spin évolue comme si les offsets des noyaux $I$ et $S$ sont nuls,
puisque leur action est en annulée (refocalisée) à chaque instant.
L'hamiltonien d'évolution est alors
\begin{eqnarray}
H & = & 2\pi J \boldsymbol{\vec{I}} \cdot \boldsymbol{\vec{S}} \\
\label{eqn:hisotrope}
& = & \pi J (2I_xS_x + 2I_yS_y + 2I_zS_z)
\end{eqnarray}
à comparer avec \ref{eqn:hcouplefort}.
L'expression de $H$ ne fait intervenir qu'un produit scalaire d'opérateurs,
invariant par permutation des axes, faisant ainsi que
$H$ soit qualifié d'\emph{isotrope}.

L'évolution de l'aimantation d'un des noyaux, $I$ par exemple,
sous l'action de $H$ peut être prédite de diverses manières.
Celle présentée ici essaye de réutiliser au maximum les équations
écrites dans les paragraphes précédents.
La séquence de calculs adoptée est la suivante :
\begin{equation}
\sigma_0 = I_{\pm}\flham{a H_2 t} \sigma_1
\flham{a H_1 t} \sigma_2
\end{equation}
où
\begin{eqnarray}
H & = & a \, H_1 + a \, H_2 \\
a & = & \pi J \\
H_1 & = & 2I_zS_z \\
H_2 & = & 2I_xS_x + 2I_yS_y
\end{eqnarray}

Sachant que
\begin{equation}
[H_2, I_{\pm}] = \mp 2I_zS_- \qetq [H_2, 2I_zS_{\pm}] = \mp I_{\pm}
\end{equation}
on déduit :
\begin{equation}
\sigma_1 = \cos at \, I_{\pm} \pm i \sin at \, 2I_zS_{\pm}
\end{equation}

De même, à partir de :

\noindent\hfill
\parbox{6cm}{
\[
\begin{array}{r@{\,=\,}l@{\quad\quad}r@{\,=\,}l}
[H_1, I_{\pm}] & \pm2I_-S_z & [H_1, 2I_{\pm}S_z] & \pm I_- \\
{[H_1, 2I_zS_{\pm}]} & \pm S_{\pm} & [H_1, S_{\pm}] & \pm 2I_zS_{\pm}
\end{array}
\]
}
\hfill
\parbox{1cm}{
\begin{eqnarray}\end{eqnarray}
}

\noindent
il vient :
\begin{eqnarray}
I_{\pm} & \flham{a H_1 t} &
\cos at \, I_{\pm} \mp i \sin at \, 2I_{\pm}S_z \\
2I_zS_{\pm} & \flham{a H_1 t} &
\cos at \, 2I_zS_{\pm} \mp i \sin at \, S_{\pm}
\end{eqnarray}
et donc
\begin{eqnarray}
\lefteqn{\sigma_2 = \cos at (\cos at I_{\pm} \, \mp i \sin at \, 2I_{\pm}S_z)}
\nonumber\\
& & \quad\quad \pm i \sin at (\cos at \, 2I_zS_{\pm} \, \mp i \sin at \, 2S_{\pm})
\end{eqnarray}
soit, en calculant la demi-somme des équations d'évolution de $I_+$ et de $I_-$ :
\begin{equation}
I_x \flham{H t} \sigma_x(t) = \cos^2 at \, I_x + \sin^2 at \, S_x
+ \sin at \cos at (2I_yS_z - 2I_zS_y)
\end{equation}
ou encore
\begin{equation}
\label{eqn:isox}
\sigma_x(t) = \frac{I_x + S_x}{2}
+ \cos(2\pi J  t) \frac{I_x - S_x}{2} + \frac{1}{2} \sin(2\pi J t) \, (2I_yS_z - 2I_zS_y)
\end{equation}

Ainsi, après un temps $\tau=1/(2J)$ d'évolution, l'aimantation transversale (en phase)
$I_x$ du noyau $I$ est intégralement convertie en aimantation transversale (en phase) $S_x$ du
noyau $S$ et revient à l'état initial après une seconde période de durée $\tau$.
Notons qu'en dehors des instants où l'aimantation est intégralement sur
$I$ ou sur $S$, des états anti-phase, d'intensité relative maximale $1/2$, sont aussi créés.
Le processus de transfert d'aimantation entre noyaux qui vient d'être décrit porte
le nom de \emph{mélange isotrope}, à cause de la nature de l'hamiltonien
qui en est la cause.
Le mélange isotrope constitue le fondement de l'expérience TOCSY.

L'hamiltonien \ref{eqn:hisotrope} préserve l'ordre de cohérence total des états
qui sont mélangés : $I_-$, $S_-$, $2I_-S_z$ et $2I_zS_-$.
Le calcul de l'évolution de $I_x$ peut aussi être mené en n'utilisant que les règles
d'évolution des opérateurs cartésiens, sachant que les trois termes qui
constituent l'hamitonien \ref{eqn:hisotrope} commutent entre eux :
\begin{equation}
I_x \flham{2I_xS_x t} I_x
\flham{2I_yS_y t} \quad
\flham{2I_zS_z t}
\sigma_x(t)
\end{equation}
avec bien entendu un résultat identique à \ref{eqn:isox}.
Suivant la même procédure, l'évolution de $I_y$ donne un résultat
$\sigma_y(t)$ qui se déduit de $\sigma_x(t)$ par permutation circulaire
des indices $x$, $y$ et $z$.
L'évolution de $I_{\pm}$ en $\sigma_x(t) \pm i\sigma_y(t)$ fait apparaître la
conservation de l'ordre de cohérence $\pm 1$.

\subsection{Effet d'offset}

\subsubsection{Position du problème}
Pendant une impulsion RF de phase nulle, l'hamiltonien qui agit sur un noyau $I$ isolé,
considéré dans le référentiel tournant, résulte de l'interaction due à l'offset
du noyau et au champ RF (statique dans le référentiel tournant).
La description classique des phénomènes est détaillée dans l'annexe \ref{chap:refer}.
L'usage de la matrice densité a été limité le plus souvent aux impulsions parfaites,
pour laquelle l'action du champ RF est très notablement supérieure à toutes les autres.
Les opérateurs associés aux impulsions sont alors $I_{\pm x, \pm y}$.
Dans la réalité, l'hamiltonien qui agit pendant une impulsion RF de phase nulle s'écrit
\begin{equation}
H = a \, I_z + b \, I_x \qavecq a = \omsi
\qetq b = \omuns
\end{equation}

Le traitement classique nous apprend que l'aimantation effectue une mouvement
de précession autour de la direction du champ effectif.
Le but de ce paragraphe est d'arriver à ce résultat à l'aide du formalisme
développé précédemment et aussi d'introduire le concept de sélectivité
des impulsion de RF.

\subsubsection{Solution du problème général}
Le système se trouve dans l'état initial $\sigma_0 = I_z$.
Les relations de commutation :
\begin{eqnarray}
[H, I_z] & = & -ib \, I_y \\
{[H, I_y]} & = & -ia \, I_x  + ib \, I_z \\
{[H, I_x]} & = & ia \, I_y 
\end{eqnarray}
indiquent que le problème qui est traité est d'ordre 3.
Tous les problèmes pour lesquelles les relations de commutation
ont la même structure sont traitables à l'aide des considérations qui vont suivre.
On peut citer dans cette catégorie l'action d'un champ RF en résonance
sur un noyau couplé.

Il faut en fait trouver trois combinaisons de $I_x$, $I_y$ et $I_z$
pour lesquelles les relations de commutation se ramènent aux cas précédemment étudiés.
De manière immédiate :
\begin{eqnarray}
[H, a \, I_z + b \, I_x] = 0
\end{eqnarray}
ce qui signifie que $a \, I_z + b \, I_x$ ou tout multiple de celui-ci est
invariant par action de $H$.
Ainsi la direction du vecteur $\aimvec = a\kprimvec + b\iprimvec$ joue
le rôle de la direction du champ effectif.
L'aimantation décrite par $I_y$ est perpendiculaire à celle décrite par
$a \, I_z + b \, I_x$ car le produit scalaire des vecteurs correspondants est nul.
L'évolution de $I_y$ est gouvernée par les relations :
\begin{eqnarray}
{[H, I_y]} & = &i(-a \, I_x  + b \, I_z) \\
{[H, -a \, I_x  + b \, I_z]} & = & -i (a^2 + b^2) I_y
\end{eqnarray}
En introduisant les notations :
\begin{eqnarray}
c & = & \sqrt{a^2 + b^2} = \sqrt{\omuns^2 + \omsi^2} = \omeff \\
U & = & \frac{a \, I_z + b \, I_z}{c} = \ct \, I_x + \st \, I_z \\
V & = & I_y \\
W & = & \frac{-a \, I_x + b \, I_z}{c} = -\st \, I_x + \ct \, I_z
\end{eqnarray}
les realtions de commutation deviennent
\begin{eqnarray}
[H,U] & = & 0 \\
{[H,V]} & = & ic \, W \\
{[H,W]} & = & -ic \, V
\end{eqnarray}
ramenant ainsi un problème d'ordre 3 à deux problèmes, l'un d'ordre 0
et l'autre d'ordre 2.
Les équations d'évolution s'écrivent alors :
\begin{eqnarray}
U & \flham{Ht} & U \\
V & \flham{Ht} & \cos ct \, V + \sin ct \, W \\
W & \flham{Ht} & -\sin ct \, V + \cos ct \, W
\end{eqnarray}
Ce qui correspond bien à une rotation de l'aimantation autour de la direction de
$U$ (par abus de langage, il s'agit en fait de la direction de l'aimantation décrite par $U$)
à la pulsation $\omeff$.

Pour déterminer le devenir d'un vecteur aimantation initial quelconque
il faut l'écrire dans la base \{$U$, $V$, $W$\}, utiliser les équations
d'évolution ci-dessus puis réécrire l'état $\sigma(t)$
du système au temps $t$ dans la base \{$I_x$, $I_y$, $I_z$\}.
Ainsi :
\begin{equation}
xI_x + yI_y + zI_z = x'U + y'V +z'W \flham{Ht}
x''U + y''V + z''W = x'''I_x + y'''I_y + z'''I_z
\end{equation}
En considérant chacun des triplets de facteurs multiplicatifs comme les
coordonnées cartésiennes de vecteurs dans l'espace physique
muni du repère $(\iprimvec, \jprimvec, \kprimvec)$, l'opération
qui transforme $(x',y',z')$ en $(x'',y'',z'')$ est, d'après ce qui précède,
la rotation d'angle $\omeff t$ autour de l'axe dirigé par $\iprimvec$.

Le passage de $(x'',y'',z'')$ vers $(x''',y''',z''')$ est explicité par :
\begin{eqnarray}
\lefteqn{x'''I_x + y'''I_y + z'''I_z } \nonumber\\
& = & x''U + y''V + z''W  \\
& = & x''(\ct \, I_x + \st \, I_z) + y''I_y + z''(-\st \, I_x + \ct \, I_z) \\
& = & (x''\ct - z''\st)I_x + y''I_y + (x''\st + z''\ct) I_z
\end{eqnarray}
qui définit la transformations des vecteurs de base de l'espace physique :
\begin{eqnarray}
(x'',y'',z'') = (1,0,0) & \longrightarrow & (\ct, 0, \st) = (x''', y''', z''') \\
(0,1,0) & \longrightarrow & (0, 1, 0) \\
(0,0,1) & \longrightarrow & (-\st, 0, \ct)
\end{eqnarray}
et qui correspond donc à une rotation d'angle $\theta$ dans le plan $(\iprimvec, \kprimvec)$
autour de l'axe défini par $\jprimvec$.

Le passage de $(x,y,z)$ vers $(x',y',z')$ est l'opération inverse de celui de
$(x'',y'',z'')$ vers $(x''',y''',z''')$ et correspond donc à
une rotation d'angle $-\theta$ dans le plan $(\iprimvec, \kprimvec)$
autour de l'axe défini par $\jprimvec$.
Cette rotation transforme la direction du champ effectif donnée par les coordonnées
$(\ct, 0, \st)$ en $(1,0,0)$.

En résumé, la rotation d'angle $\omeff t$ autour de l'axe du champ effectif est identique
à l'application successive de trois opérations :
\begin{enumerate}
\item Une rotation d'angle $-\theta$ autour de l'axe OY, rotation qui amène le champ effectif
sur l'axe OX
\item Une rotation autour de OX d'angle $\omeff t$
\item Une rotation d'angle $\theta$ qui ramène le champ effectif à sa position d'origine.
\end{enumerate}
Toute rotation autour d'un axe quelconque contenu dans un plan défini par deux
axes du référentiel
est équivalente à la composition de 3 rotations autour des axes de ce référentiel.
Si l'axe de rotation est défini par deux angles polaires $\theta$ et $\phi$ quelconques
(la phase de l'impulsion n'est pas nécessairement nulle ou multiple de $\pi/2$),
alors, il faut en tout d'abord ramener l'axe de rotation dans le plan $XOZ$ par une rotation
d'angle $-\phi$ autour de $OZ$, puis appliquer les trois rotations mentionnées ci-dessus,
et enfin appliquer une rotation d'angle $\phi$ autour de $OZ$ pour ramener l'axe à sa
position initiale. Cette composition de cinq rotations se résume par :
\begin{equation}
\label{eqn:cinqrotations}
R_{\alpha}(\theta, \phi) = 
R_{\phi}(\pi/2, 0)
R_{\theta}(0, \pi/2)
R_{\alpha}(0,0)
R_{-\theta}(0, \pi/2)
R_{-\phi}(\pi/2, 0)
\end{equation}
où $R_{\alpha}(\theta, \phi)$ désigne la rotation d'angle $\alpha$ autour d'un axe
défini par les angles $\theta$ (compté à partir du plan $XOY$) et $\phi$ (compté à partir
du plan $XOZ$). 
Dans le membre de droite de l'équation \ref{eqn:cinqrotations},
la première opération à appliquer est celle la plus à droite ;
sa réécriture en terme de superopérateurs est
\begin{equation}
\label{eqn:cinqrotationsop}
\flham{-\phi I_z}
\quad
\flham{\theta I_y}
\quad
\flham{\alpha I_x}
\quad
\flham{-\theta I_y}
\quad
\flham{-\phi I_z}
\end{equation}
L'angle $\theta$ a été transformé en son opposé car c'est l'opérateur
$\theta I_y$ qui transforme $\cos\theta I_x + \sin\theta I_y$ (direction
du champ effectif) en $I_x$ (axe pour la rotation d'angle $\alpha$).
Ce changement de signe est dû au fait que le trièdre $(\ivec, \kvec, \jvec)$
est d'orientation opposée à $(\ivec, \jvec, \kvec)$.

\subsubsection{Lien avec l'écho de spin}
Un cas particulier intéressant concerne la définition de ce qu'est, par exemple,
une rotation d'angle $-\alpha$ autour de l'axe $0Z$ ($\theta=\pi/2$, $\phi=0$).
On peut considérer qu'il s'agit d'une rotation d'angle $\alpha$ autour de l'axe $OZ$
qui aurait été préalablement et subséquemment tourné de $\pi$ autour d'un axe horizontal
quelconque defini par ($\theta=0$, $\phi$). Ainsi :
\begin{equation}
\label{eqn:rotationpi}
R_{-\alpha}(\pi/2, 0)
= 
R_{\pi}(0, \phi)
R_{\alpha}(\pi/2,0)
R_{\pi}(0, \phi)
\end{equation}
soit encore :
\begin{equation}
\label{eqn:rotationpi2}
R_{\pi}(0, \phi)
R_{\alpha}(\pi/2, 0)
=
R_{-\alpha}(\pi/2,0)
R_{\pi}(0, \phi)
\end{equation}
qui se traduit en termes de superopérateurs par
\begin{equation}
\label{eqn:rotationpiop}
\flham{\alpha I_z}
\quad
\flham{\pi I_{x,y}}
\qiffq
\flham{\pi I_{x,y}}
\quad
\flham{-\alpha I_z}
\end{equation}

L'équivalence \ref{eqn:rotationpiop}
entre successions d'opérateurs est à base de la justification 
de la construction de l'hamiltonien réduit associé aux séquences d'écho de spin.
Pour un système à un spin, on posant $\alpha = \omsi T$, les équivalences :
\begin{eqnarray}
\flham{\alpha I_z}
\quad
\flham{\pi I_{x,y}}
\quad
\flham{\alpha I_z}
& \qiffq &
\flham{\pi I_{x,y}}
\quad
\flham{-\alpha I_z}
\quad
\flham{\alpha I_z} \\
& \qiffq &
\flham{\pi I_{x,y}}
\end{eqnarray}
montrent bien que l'hamiltonien réduit est nul et que seule
subsiste l'action de l'impulsion de refocalisation.

L'extension aux systèmes à plusieurs spins est fondée sur une
généralisation de \ref{eqn:rotationpiop}, à savoir :
\begin{equation}
\label{eqn:rotationpiopzz}
\flham{\alpha 2I_zS_z}
\quad
\flham{\pi I,S_{x,y}}
\qiffq
\flham{\pi I,S_{x,y}}
\quad
\flham{-\alpha 2I_zS_z}
\end{equation}

\subsubsection{Sélectivité des impulsions}

Le but est ici de déterminer les composantes de l'aimantation
en fonction de l'offset de l'impulsion qui lui est appliquée
sachant d'une part que l'aimantation initiale est celle d'équilibre
et que l'angle de nutation $\alpha_0$ à offset nul est constant.
La phase de l'impulsion sera $\phi = \pi/2$, de manière à ce que si $\alpha_0 = \pi/2$
alors l'aimantation finale sera mesurée positivement sur l'axe $OX$.

Pour cela, il sera fait appel aux transformations de 
l'identité \ref{eqn:cinqrotationsop} :
\begin{equation}
\sigma_0 = I_z
\stackrel{-\pi/2 I_z}{\Longrightarrow} I_z
\flham{\theta I_y} \sigma_1
\flham{\alpha I_x} \sigma_2
\flham{-\theta I_y} \sigma_3
\flham{\pi/2 I_z} \sigma_4
\end{equation}
qui fournissent :
\begin{eqnarray}
\sigma_1 & = & \ct I_z + \st I_x \\
\sigma_2 & = & \ct(\ca I_z - \sa I_y) + \st I_x \\
\sigma_2 & = & \st I_x - \ct\sa I_y + \ct\ca I_z \\
\sigma_3 & = & \st(\ct I_x + \st I_z) - \sa\ct I_y \nonumber\\
& & \quad + \ct\ca(\ct I_z - \st I_x) \\
\sigma_3 & = & \st\ct(1 - \ca) I_x - \sa\ct I_y \nonumber\\
& & \quad +(\sin^2\theta + \ca\cos^2\theta)I_z \\
\sigma_4 & = & \sa\ct I_x + \st\ct(1 - \ca) I_y \nonumber\\
& & \quad +(\sin^2\theta + \ca\cos^2\theta)I_z
\end{eqnarray}

On pose alors :
\begin{eqnarray}
v & = & \omsi / \omuns \\
\omeff & = & \omuns \sqrt{1 + v^2} \\
\alpha & = & \alpha_0 \sqrt{1 + v^2} \\
\ct & = & \frac{v}{\sqrt{1+v^2}} \\
\st & = & \frac{1}{\sqrt{1+v^2}}
\end{eqnarray}
La variable sans dimension $v$ caractérise l'offset en prenant $\omuns$
comme unité de mesure.
La sélectivité de l'impulsion est mesurée par les fonctions de la variable $v$ :
\begin{eqnarray}
\frac{\aimxs}{\aimzerozs} & = &
\frac{\sin(\alpha_0\sqrt{1+v^2})}{\sqrt{1+v^2}} \\
\frac{\aimys}{\aimzerozs} & = &
\frac{v}{1+v^2} \left( 1 - \cos(\alpha_0\sqrt{1+v^2})\right)\\
\frac{\aimzs}{\aimzerozs} & = &
\frac{\cos(\alpha_0\sqrt{1+v^2}) + v^2}{1+v^2}
\end{eqnarray}
dont les représentations graphiques pour $\alpha_0 = \pi/2$ et 
$\alpha_0 = \pi$ sont données par la figure \ref{fig:sqrselect}.

\begin{figure}[hbt]
\begin{center}
\begin{pspicture}(-7.5,-2)(7.5,2)
\rput(-4.5,0){
\psset{xunit=0.6cm,yunit=2cm}
\psaxes[Dx=5](0,0)(-5,-1)(5,1)
\fileplot{sqr90zx}
\fileplot[linestyle=dashed,dash=2pt 5pt]{sqr90zy}
\fileplot[linestyle=dashed,dash=5pt 5pt]{sqr90zz}
}
\rput(2,0){
\psset{xunit=0.6cm,yunit=2cm}
\psaxes[Dx=5](0,0)(-5,-1)(5,1)
\fileplot{sqr180zx}
\fileplot[linestyle=dashed,dash=2pt 5pt]{sqr180zy}
\fileplot[linestyle=dashed,dash=5pt 5pt]{sqr180zz}
}
\rput(6,1.5) {
\rput(0,0){$\displaystyle\frac{\displaystyle\aimxs}{\displaystyle\aimzerozs}$}
\psline[linewidth=0.04](0.5,0)(1.5,0)
}
\rput(6,0) {
\rput(0,0){$\displaystyle\frac{\displaystyle\aimys}{\displaystyle\aimzerozs}$}
\psline[linewidth=0.04,linestyle=dashed,dash=2pt 5pt](0.5,0)(1.5,0)
}
\rput(6,-1.5) {
\rput(0,0){$\displaystyle\frac{\displaystyle\aimzs}{\displaystyle\aimzerozs}$}
\psline[linewidth=0.04,linestyle=dashed,dash=5pt 5pt](0.5,0)(1.5,0)
}
\end{pspicture}
\caption{\label{fig:sqrselect}
\small Sélectivité d'une impulsion rectangulaire, à gauche d'angle $\pi/2$,
à droite d'angle $\pi$}
\end{center}
\end{figure}

\subsubsection{Rotation générale dans l'espace physique}
$R(\rvec)$ désigne ici le vecteur produit par une
rotation d'angle $\alpha$ appliquée au vecteur $\rvec$.
Le sens et la direction de l'axe de rotation sont définis 
par un vecteur unitaire $\uvec$.
Le vecteur $\rvec$ se décompose en deux vecteurs $\rvecpar$ et $\rvecper$
respectivement parallèles et perpendiculaires à $\uvec$ :
\begin{equation}
\rvec = \rvecpar + \rvecper
\end{equation}
La composante parallèle à $\uvec$ est la projection de $\rvec$ sur $\uvec$.
En considérant que $\uvec$ est un vecteur unitaire :
\begin{eqnarray}
\rvecpar & = & (\uvec \cdot \rvec)\uvec \\
\rvecper & = & \rvec - (\uvec \cdot \rvec)\uvec
\end{eqnarray}
Par linéarité de l'opération de rotation :
\begin{equation}
R(\rvec) = R(\rvecpar) + R(\rvecper)
\end{equation}
Par définition de $\rvecpar$, $R(\rvecpar) = \rvecpar$.
Le vecteur $\rvecper$ tourne d'un angle $\alpha$ dans le plan défini par
lui-même et $\uvec \wedge \rvecper$.
Sachant que $\uvec \wedge \rvecper = \uvec \wedge \rvec$
car $\uvec \wedge (\uvec \cdot \rvec)\uvec = 0$,
\begin{equation}
R(\rvecper) = \ca \rvecper + \sa \uvec \wedge \rvec
\end{equation}
et donc
\begin{eqnarray}
R(\rvec) & = & (\uvec \cdot \rvec)\uvec + \ca (\rvec -(\uvec \cdot \rvec)\uvec)
+ \sa (\uvec \wedge \rvec) \\
\label{eqn:rotatgene}
& = & \ca \rvec + \sa (\uvec \wedge \rvec) + (1 - \ca)(\uvec \cdot \rvec)\uvec
\end{eqnarray}

La relation entre $\rvec$ et $R(\rvec)$ est linéaire.
Il suffit de savoir comment sont transformés les vecteurs de base
de l'espace physique pour savoir comment est transformé tout vecteur.
L'équation \ref{eqn:rotatgene} est à la base de la prédiction numérique
des fonctions de sélectivité des impulsions "formées" (shaped pulses, en anglais)
pendant lesquelles $\omuns$ voire $\phi$ varient.
