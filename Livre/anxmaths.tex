\chapter{Un peu de maths...}
\label{chap:maths}

\section{Nombres complexes}
\subsection{Écritures et définitions}
Un nombre complexe $z$ est défini par
\begin{equation}
\label{eqn:defcomplexe}
z = a + ib \qavecq i^2 = -1 \qetq a,\,b 
\;\;\mbox{deux nombres réels}
\end{equation}

Cette définition mériterait une digression
philosophico--mathématique, mais ce n'est pas ici le lieu pour le faire.
La définition \ref{eqn:defcomplexe} peut se réécrire :
\begin{eqnarray}
z & = & \sqrt{a^2+b^2}
\left( \frac{a}{\sqrt{a^2+b^2}} + i\frac{b}{\sqrt{a^2+b^2}} \right) \nonumber\\
\label{eqn:zexpcomplexe}
& = & |z| \left( \cos\theta + i\sin\theta \right)
\end{eqnarray}

Les quantités réelles $|z|$ et $\theta$ sont respectivement nommées module
et argument du nombre complexe $z$.
La dérivée de la fonction $f$ de la variable $\theta$ définie par
\begin{equation}
\label{eqn:defexpcomplexe}
f(\theta) = \cos\theta + i\sin\theta
\end{equation}
possède une propriété intéressante :
\begin{eqnarray}
f'(\theta) & = & -\sin\theta +i\cos\theta \nonumber\\
& = & i(\cos\theta + i\sin\theta) \nonumber\\
& = & i f(\theta)
\end{eqnarray}
Trouver la fonction $f(\theta)$ qui possède cette propriété et qui satisfasse
la relation \ref{eqn:defexpcomplexe} revient à résoudre l'équation différentielle
\begin{equation}
\derivtot{f}{\theta} = i f.
\end{equation}
Ainsi
\begin{eqnarray}
\frac{\diffe f}{f} & = & i \diffe\theta \nonumber\\
\diffe\log(f) & = & \diffe(i\theta) \nonumber\\
\log(f) = i\theta + K 
\end{eqnarray}
La constante $K$ vaut 0 car $ \log(f(0)) = 0 $ et donc
\begin{equation}
\label{eqn:expcomplexe}
\cos\theta + i\sin\theta = e^{i\theta} = \exp(i\theta)
\end{equation}
Ce résultat, combiné avec l'équation \ref{eqn:zexpcomplexe}
conduit à l'écriture d'un nombre complexe à partir de son
module et de son argument :
\begin{equation}
z = |z| e^{i\theta}
\end{equation}
aussi appelée écriture "trigonométrique" ou "exponentielle complexe".

Le complexe conjugué $\bar{z}$ d'un nombre complexe $z = a + ib$ 
est défini par :
\begin{equation}
\bar{z} = a - ib
\end{equation}
avec comme écriture trigonométrique
\begin{equation}
\bar{z} = |z| e^{-i\theta}
\end{equation}

\subsection{Opérations entre nombres complexes}
Considérons deux nombres complexes $z_1$ et $z_2$
\begin{eqnarray}
z_1 = a_1 + i b_1 & \quad & z_1 = |z_1| e^{i\theta_1} \nonumber\\
z_2 = a_2 + i b_2 & \quad & z_2 = |z_2| e^{i\theta_2} \nonumber
\end{eqnarray}
Les opérations entre nombres complexes sont définies par
\begin{eqnarray}
z_1 + z_2 & = & (a_1 + a_2) + i (b_1 + b_2) \\
z_1 - z_2 & = & (a_1 - a_2) + i (b_1 - b_2) \\
z_1 z_2 & = & (a_1 a_2 - b_1 b_2) + i (a_1 b_2 + a_2 b_1) \\
z_1 z_2 & = & |z_1||z_2| \exp(i(\theta_1 + \theta_2)) \\
z_1 \bar{z_1} & = & |z_1|^2 \\
\frac{1}{z_1} & = & \frac{1}{|z_1|} \exp(-i\theta) = \frac{\bar{z_1}}{|z_1|^2} \\
\frac{z_1}{z_2} & = & \frac{|z_1|}{|z_2|} \exp(i(\theta_1 - \theta_2)) = \frac{z_1 \bar{z_2}}{|z_2|^2} \\
\frac{z_1}{z_2} & = & \frac{(a_1 a_2 + b_1 b_2) + i(a_2 b_1 - a_1 b_2)}{a_2^2 + b_2^2}
\end{eqnarray}

L'identification entre le plan vectoriel et l'ensemble des nombres complexes
est illustré par la figure \ref{fig:xyplane}, page \pageref{fig:xyplane}.
L'addition et soustraction des nombres complexes est équivalente aux mêmes
opérations effectuées sur les vecteurs. Le module d'un nombre complexe est la norme
du vecteur associé, son argument est l'angle formé entre le vecteur et le vecteur
$\ivec$. La transformation d'un nombre complexe en son complexe conjugué correspond
à l'opération de symétrie par rapport à l'axe $OX$ de direction $\ivec$.

La multiplication d'un nombre complexe par un autre nombre complexe de module 1
et d'argument $\theta$ est équivalente à une rotation d'angle $\theta$ :
\begin{equation}
z_1 \exp(i\theta) = |z_1| \exp(i(\theta_1 + \theta))
\end{equation}
Le module de $z_1$ est inchangé et son argument a été augmenté de $\theta$.
La description d'une opération de rotation plane est donc très simplement
décrite en utilisant les nombre complexes.

