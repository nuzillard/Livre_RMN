\chapter[RMN 1D]{Expériences à une dimension}

La plus simple des expériences à une dimension (1D) est celle où après
une période de mise ou de remise en équilibre de l'échantillon, 
l'excitation des systèmes de spins par une unique impulsion 
de radiofréquence est suivie de la détection du signal de précession libre.
Le terme "1D" est relatif au fait que le spectre obtenu après TF
est une fonction d'une seule variable fréquencielle.
Des expériences 1D plus complexes font intervenir la relaxation,
divers types de transfert d'aimantation, la spectroscopie à plusieurs quanta
ou le découplage pour améliorer ou compléter les informations contenues dans
les spectres "élémentaires".
Les mesures des temps de relaxation et d'effet Overhauser nucléaire précisent 
la nature de l'environnement des noyaux à travers l'espace.
Les liaisons chimiques sont le support du couplage scalaire et autorisent
l'analyse des systèmes de spin grâce aux transferts d'aimantation. 
Enfin, le découplage est une méthode expérimentale indispensable
en RMN hétéronucléaire car elle apporte à la fois une simplification
des spectres et une amélioration de la sensibilité.
Des applications de ces concepts à la spectroscopie 1D hétéronucléaire
et aux expériences homonucléaires sélectives 1D seront présentées ci-après.

\section{Effet Overhauser Nucléaire, nOe}
\label{sec:noe}
Le problème le plus critique posé par l'enregistrement de spectres 
de noyaux isotopiquement dilués et de faible rapport gyromagnétique 
est l'obtention d'un rapport signal sur bruit suffisant pour permettre
une l'exploitation analytique des données.
C'est par exemple le cas des noyaux de \carb.

L'effet Overhauser nucléaire se manifeste par une modification des 
intensités de certaines résonances lorsque d'autres sont saturées.
Cet effet se manifeste entre noyaux de nature semblable ou différente,
ainsi qu'entre noyaux et électrons (effet Overhauser électronique),
pour peu qu'un couplage dipolaire existe entre les spins des particules.

Le couplage dipolaire est l'interaction directe qui s'exerce à travers
l'espace entre deux particules qui possèdent un moment magnétique.
En considérant ces deux particules dans un état quantique défini par
les deux valeurs de $m_s$, l'énergie de l'interaction dépend de l'orientation
de la droite qui joint les particules avec la direction de $\bzerovec$.
Au cours du temps, la réorientation aléatoire des molécules
au sein d'un liquide isotrope produit une interaction 
dont l'intensité est nulle en moyenne. 
Le couplage dipolaire est ainsi indétectable si on ne considère que
les fréquences de résonance des noyaux.
Toutefois, l'existence d'une interaction d'intensité aléatoire
entre noyaux contribue à la relaxation de leur aimantation macroscopique
(il faut, à ce niveau de l'exposé, l'admettre).
De fait, toute perturbation aléatoire des niveaux énergétiques des noyaux
contribue à la relaxation.
L'existence du couplage dipolaire a pour conséquence que la relaxation
longitudinale de chaque noyau est influencée par celle de l'autre noyau.

Considérons deux noyaux $I$ et $S$ de rapports gyromagnétiques $\gi$ et $\gs$.
Les différences de population $\Delta P(I)$ et $\Delta P(S)$ sont
aussi appelées polarisations et notées $\poli$ et $\pols$, dont les
valeurs d'équilibre sont $\poliz$ et $\polsz$.
L'évolution des polarisations est donnée par les équations de Solomon :
\begin{eqnarray}
\label{eqn:soli}
\derivt{\poli} & = & -\rho_I \dpoli -\sigma \dpols \\
\label{eqn:sols}
\derivt{\pols} & = & -\rho_S \dpols -\sigma \dpoli
\end{eqnarray}
qui constituent une généralisation de l'équation $\ref{eqn:t1}$
relative à la relaxation longitudinale.

La grandeur $\sigma$ caractérise la vitesse (en $s^{-1}$) de relaxation \emph{croisée}
entre les noyaux $I$ et $S$.
Dans le cas où $\sigma$ est nul, les aimantations de $I$ et $S$ évoluent indépendamment.
Ainsi
\begin{equation}
\frac{1}{T_1(I)} = \rho_I^{\dipolaire} + \rho_I^{\autres}
\end{equation}
En effet, la  relaxation longitudinale de $I$ est en partie causée par son interaction
dipolaire avec $S$, mais aussi par d'autres influences de nature aléatoire.

La saturation de l'aimantation du noyau $S$ par un champ de radiofréquence continu
(voir section \ref{sec:eqnbloch}) d'intensité suffisante conduit à $\pols = 0$
et donc à une évolution des polarisations jusqu'à un nouvel état stationnaire
différent de l'état d'équilibre.
A l'état stationnaire, l'équation \ref{eqn:soli} conduit à
\begin{eqnarray}
0 & = & -\rho_I \dpoli + \sigma \polsz \\
\frac{\dpoli}{\poliz} & = & \frac{\sigma}{\rho_I} \frac{\polsz}{\poliz}
\end{eqnarray}
Sachant d'après l'équation \ref{eqn:diffpopgamma} que les polarisations 
d'équilibre sont proportionnelles
aux rapports gyromagnétiques, le rapport d'effet Overhauser $\eta$ s'écrit :
\begin{equation}
\eta = \frac{\dpoli}{\poliz} = \frac{\gs}{\gi} \frac{\sigma}{\rho_I}
\end{equation}
Il caractérise l'augmentation relative d'intensité du signal de $I$ lorsque l'aimantation
des noyaux $S$ est préalablement saturée puisque pour un noyau donné,
l'intensité des signaux mesurés est proportionnelle aux différences de population.
Dans les situations où le couplage dipolaire est le mécanisme de relaxation prépondérant
et que les molécules se réorientent très rapidement (petites molécules en solution
de faible viscosité), le rapport $\sigma/\rho_I$ vaut $1/2$, et donc
\begin{equation}
\label{eqn:noe}
\frac{\aimzI}{\aimzzi} = 1+\eta = 1+\frac{1}{2}\frac{\gs}{\gi}
\end{equation}

\section{Découplage}
Considérons un système hétéronucléaire de deux noyaux $I$ et $S$ couplés scalairement.
L'évolution d'un état $I_{x,y}$ conduit, pendant la détection de l'aimantation de $I$, à
deux raies de résonance aux fréquences $\omsi \pm \pi J$ et d'intensité moitié par
rapport à un noyau non couplé.

Si les noyaux $S$ subissent en permanence une série d'échos de spin, 
l'hamiltonien effectif d'évolution, $\omsi I_z$, ne fait pas intervenir la constante
de couplage. 
L'aimantation de $I$ évolue alors comme si $I$ n'était pas couplé à $S$.
L'évolution libre d'un terme $2I_{x,y}S_z$ ne peut alors pas conduire
à tes termes $I_{x,y}$ et n'est donc pas susceptible de produire un signal.

Dans la pratique, la série d'échos de spin pourrait être constituée d'une 
série d'impulsions à 180 degrés sans délai intercalé. 
Sous cette forme, découpler serait équivalent à envoyer
un champ de radiofréquence continu. Si on désire que le découplage soit uniforme
pour l'ensemble des noyaux $I$ dont la fréquence de résonance s'étend
sur plusieurs (ou dizaines de) kHz, cela implique des valeurs de champ 
$\bunmax$ incompatibles avec les contraintes matérielles liées à la sonde 
ou à l'échantillon (ou aux deux) par suite soit de l'effet Joule
soit des pertes diélectriques.
Les séquences utilisées pour le découplage \emph{large bande}
sont constituée d'impulsions modulées en phase, en durée, voire en amplitude,
le plus souvent optimisées par calcul numérique.

Notons qu'une telle séquence, appliquée sur l'aimantation d'équilibre de $S$
conduit aussi à la saturation de cette aimantation.

\section{RMN du \carb}
L'enregistrement d'un spectre de RMN du \carb (noyaux $I$), 
dans sa forme la plus simple, requiert la saturation de l'aimantation des \prot
(noyaux $S$) et l'obtention d'un état stationnaire pour l'aimantation des noyaux $I$,
l'excitation de l'aimantation des noyaux de \carb, puis la détection du signal
avec application d'une séquence de découplage sur les noyaux \prot.
En résumé, il s'agit d'une séquence relaxation -- impulsion -- acquisition analogue
à celle étudiée jusqu'ici sauf que les noyaux non observés sont soumis en permanence
à une séquence de découplage.

Sauf superposition accidentelle ou liée à la symétrie des molécules étudiés, chaque
atome de carbone fournit une résonance identifiable par le déplacement chimique associé.
L'information de couplage avec les noyaux \prot est perdue, mais en contrepartie
les enchevêtrements de multiplets qui rendent les spectres peu lisibles sont éliminés.
L'intensité des signaux des \carb liés à des \prot est multipliée par un facteur pouvant aller
jusqu'à 3 ($\gs/\gi = 4$ dans l'équation \ref{eqn:noe}).
Les signaux des carbones quaternaires ne profitent évidemment pas (ou peu) de l'effet Overhauser.
De plus, leur temps de relaxation peut valoir plusieurs secondes car ils ne sont
que faiblement soumis à la relaxation dipolaire.
Le temps qu'il faudrait attendre entre deux impulsions (temps de répétition $T_R$)
pour que l'aimantation de $I$
revienne près de sa valeur initiale rend peu efficace l'augmentation du rapport 
signal sur bruit par accumulation de plusieurs \FID.
Cela conduit à utiliser un angle de nutation de l'aimantation optimisé par rapport 
à $T_R$ et à $T_1$, angle qui peut être très inférieur à $\pi/2$ et
qui dépend de $T_1$, grandeur qui est inconnue le plus souvent.
La saturation partielle des signaux de \carb et l'inégalité des noyaux devant
l'effet Overhauser rend hasardeuses les tentatives de tirer une information quantitative
des spectres enregistrées dans des conditions inappropriées.

L'enregistrement de spectres de RMN du \carb dans des conditions quantitatives
requiert de n'actionner le découplage que pendant l'acquisition du signal
et d'utiliser le $T_R$ le plus long possible compatible avec le rapport signal sur bruit
désiré et le temps total de mesure disponible.

L'augmentation de l'intensité des signaux des noyaux $S$ peut aussi être réalisée
sans recourir à l'effet Overhauser, comme indiqué ci-après.

\section{Transfert hétéronucléaire d'aimantation}
Dans la suite du texte, et contrairement à ce qui est écrit depuis le début
de ce chapitre, les noyaux \prot seront notés $I$ et les hétéronoyaux $S$,
indépendemment de la nature du noyau détecté.
Dans l'étude de l'effet Overhauser, il est d'usage d'appeler $S$
le noyau dont l'aimantation est \emph{s}aturée.

\begin{figure}[hbt]
\begin{center}
\begin{pspicture}(0,0)(6,5.5)
% sequence I
\psline(1,3.5)(6,3.5)
\rput(0.5,4){RF($I$)}
\psline[linewidth=2mm]{-}(1.9,3.5)(1.9,4.5)
\rput(1.9,4.7){$\pi/2_x$}
\psline[linewidth=2mm]{-}(3.9,3.5)(3.9,4.5)
\rput(3.9,4.7){$\pi/2_y$}
\rput(2.9,4){$T$}
% sequence S
\psline(1,2)(6,2)
\rput(0.5,2.5){RF($S$)}
\psline[linewidth=2mm]{-}(4.1,2)(4.1,3)
\rput(4.3,3.2){$\pi/2_x$}
% time marks
\psline{->}(1,1)(6,1)
\psline[linewidth=0.25mm,linestyle=dashed]{-}(1.8,3.3)(1.8,0.8)
\rput(1.7,0.5){0}
\psline[linewidth=0.25mm,linestyle=dashed]{-}(2,3.3)(2,0.8)
\rput(2.1,0.5){1}
\psline[linewidth=0.25mm,linestyle=dashed]{-}(3.8,3.3)(3.8,0.8)
\rput(3.7,0.5){2}
\psline[linewidth=0.25mm,linestyle=dashed]{-}(4,1.8)(4,0.8)
\rput(4,0.5){3}
\psline[linewidth=0.25mm,linestyle=dashed]{-}(4.2,1.3)(4.2,0.8)
\rput(4.3,0.5){4}
\psline[linewidth=0.25mm,linestyle=dashed]{-}(5,1.6)(5,0.8)
\rput(5,0.5){$t$}
% FID
\rput(4.2,2){
\pscurve(0,0.5)(0.5,0.25)(1.5,0)
\pscurve(0,-0.5)(0.5,-0.25)(1.5,0)
\psline(0,0.5)(0,-0.5)
}
\end{pspicture}
\caption{\label{fig:transmagheta}
Principe du transfert d'aimantation hétéronucléaire.}
\end{center}
\end{figure}

\subsection{Principe}

Détaillons l'effet de la séquence \ref{fig:transmagheta}
sur la matrice densité initiale $I_z + aS_z$ d'un système scalairement 
couplé $IS$ où $a = \gs/\gi$, soit $a = $ 0,25 si $S$ 
désigne les noyaux de \carb. 
Le temps $T$ vaut $1/2J$. 
Calculons l'évolution de $I_z$ aux instants 1 à 4 en faisant l'hypothèse
simplificatrice $\omsi = 0$ :
\begin{eqnarray}
\sigma_1 & = & - I_y \\
\sigma_2 & = & -\cos\left(\frac{\pi J}{2 J}\right)\cdot I_y + 
\sin\left(\frac{\pi J}{2 J}\right) \cdot 2I_xS_z = 2I_xS_z \\
\sigma_3 & = & -2I_zS_z \\
\label{eqn:fintransmag}
\sigma_4 & = & 2I_zS_y
\end{eqnarray}

\begin{figure}[hbt]
\begin{center}
\begin{pspicture}(0,-3)(10,2.5)
% a
\psline(0,0)(2,0)
\pstriangle*(0.5,0)(0.2,-0.5)
\pstriangle*(1.5,0)(0.2,-0.5)
\rput(0.5,-1){-1/4}
\rput(1.5,-1){-1/4}
\rput(0,-2){$a$}
% b
\psline(4,0)(6,0)
\pstriangle*(4.5,0)(0.2,2)
\pstriangle*(5.5,0)(0.2,-2)
\rput(4.5,2.5){1}
\rput(5.5,-2.5){-1}
\rput(4,-2){$b$}
% c
\psline(8,0)(10,0)
\pstriangle*(8.5,0)(0.2,1.5)
\pstriangle*(9.5,0)(0.2,-2.5)
\rput(8.5,2){3/4}
\rput(9.5,-3){-5/4}
\rput(8,-2){$c$}
% deco
\rput(3,0){+}
\rput(7,0){=}
\end{pspicture}
\caption{\label{fig:picsinept}
Intensités des pics obtenues par la séquence de la Figure \ref{fig:transmagheta}.
$a$. contribution des noyaux $S$, $b$. contribution des noyaux $I$, $c$ résultat.}
\end{center}
\end{figure}

Le spectre correspondant à l'évolution libre de $\sigma_4$ est un 
doublet antiphase en dispersion d'intensité relative 1. 
Le second terme de l'état initial, $a S_z$, commute avec tous les opérateurs 
qui interviennent entre les instants 0 et 3 et reste donc inchangé jusque là. 
L'impulsion $\pi/2$ sur $S$ le transforme en $-aS_y$ qui évolue ensuite pour 
donner un doublet en phase et en dispersion d'intensité relative $-a$.
Le spectre réellement observé, après correction de la phase 
(présentation des raies en absorption) consiste en une paire de raies 
d'intensités relatives $(1-a)$ et $(-1-a)$ soit 3/4 et -5/4 si a = 0,25
(Figure \ref{fig:picsinept}). 
En comparaison les intensités relatives des composantes du doublet qui 
aurait été obtenu par la séquence impulsion-détection, 
sans effect Overhauser ni découplage, valent $a$ et $a$. 

L'étape-clé du transfert d'aimantation est la transformation 
subie par le système sous l'action simultanée 
(si on considère des impulsions de durée pratiquement nulle)
des deux impulsions d'angle $\pi/2$ sur un état couplé du noyau $I$
qui produit un état couplé du noyau $S$.
L'intensité du signal de pulsation $\omss$ qui provient du 
transfert d'aimantation est proportionnelle à l'aimantation initiale 
du noyau $I$, plus importante que celle de $S$.

Il doit nécessairement y avoir entre les phases des deux impulsions 
appliquées aux noyaux $I$ un écart de $\pi/2$.
Si la phase de la seconde impulsion était nulle :
\begin{eqnarray}
\sigma_2 & = & 2I_xS_z \\
\sigma_3 & = & 2I_xS_z \quad \mbox{car $\sigma_2$ et $I_x$ commutent} \\
\sigma_4 & = & -2I_xS_y
\end{eqnarray}
Cet état à 0 et 2 quanta évolue sans donner de signal, ce qui n'est
pas le but recherché ici.

Dans l'expérience décrite par la figure \ref{fig:transmagheta}, 
le temps $T$ a été choisi égal à $1/2J$ pour que l'état non couplé
$\sigma_1$ n'évolue seulement que vers un état couplé. 
Dans la réalité il n'y a pas une valeur unique de $J$ mais une certaine dispersion
autour d'une valeur moyenne (environ 145 Hz, si $S$ est un noyau de \carb).
Dans le cas général $\sigma_2 = -\cos(\pi J T)I_y + \sin(\pi J T)2I_xS_z$.
Le second terme conduit au signal souhaité, mais son intensité dépend de $\sin(\pi J T)$,
qui vaut $\pm 1$ lorsque $\pi J T$ vaut $(2n+1)\pi/2$, c'est à dire lorsque $T$
est un multiple impair de $1/2J$. 
Le terme proportionnel à $I_y$ reste inchangé sous l'action de l'impulsion 
$\pi/2_y(I)$; il ne contribue donc pas au signal enregistré aux fréquences du noyau $S$.

\subsection{Lien avec le modèle vectoriel}

\begin{figure}[hbt]
\begin{center}
\begin{pspicture}(-6,-5)(6,5)
\SpecialCoor
\rput(-3,2.5){
\psline{->}(0.5;150)(2.5;-30)
\psline{->}(0.5;270)(2.5;90)
\psline{->}(0.5;30)(2.5;210)
\uput[210](2.5;210){$X$}
\uput[-30](2.5;-30){$Y$}
\uput[90](2.5;90){$Z$}
\psline[linewidth=0.08,doubleline=true,doublesep=0.05]{->}(0,0)(2;90)
\uput[90](1.8;80){$\alpha$}
\uput[90](1.8;100){$\beta$}
\rput(-0.5,0){$O$}
}
\rput(0,2.5){
\psline{->}(-1,0)(1,0)
\rput(0,0.25){$\pi/2_x(I)$}
}
\rput(3,2.5){
\psline{->}(0.5;150)(2.5;-30)
\psline{->}(0.5;270)(2.5;90)
\psline{->}(0.5;30)(2.5;210)
\uput[210](2.5;210){$X$}
\uput[-30](2.5;-30){$Y$}
\uput[90](2.5;90){$Z$}
\psline[linewidth=0.08,doubleline=true,doublesep=0.05]{->}(0,0)(2;150)
\uput[150](1.8;140){$\alpha$}
\uput[150](1.8;160){$\beta$}
\rput(0.5,0){$O$}
}
\rput(6,2.5){
\psline{->}(-1,0)(1,0)
\rput(0,0.25){$T = 1/2J$}
}
\rput(-3,-2.5){
\psline{->}(0.5;150)(2.5;-30)
\psline{->}(0.5;270)(2.5;90)
\psline{->}(0.5;30)(2.5;210)
\uput[210](2.5;210){$X$}
\uput[-30](2.5;-30){$Y$}
\uput[90](2.5;90){$Z$}
\psline[linewidth=0.08]{->}(0,0)(2;210)
\uput[210](1.8;200){$\alpha$}
\psline[linewidth=0.08]{->}(0,0)(2;30)
\uput[30](1.8;20){$\beta$}
\rput(-0.5,0){$O$}
}
\rput(0,-2.5){
\psline{->}(-1,0)(1,0)
\rput(0,0.25){$\pi/2_y(I)$}
}
\rput(3,-2.5){
\psline{->}(0.5;150)(2.5;-30)
\psline{->}(0.5;270)(2.5;90)
\psline{->}(0.5;30)(2.5;210)
\uput[210](2.5;210){$X$}
\uput[-30](2.5;-30){$Y$}
\uput[90](2.5;90){$Z$}
\psline[linewidth=0.08]{->}(0,0)(2;90)
\uput[90](1.8;80){$\alpha$}
\psline[linewidth=0.08]{->}(0,0)(2;-90)
\uput[-90](1.8;-80){$\beta$}
\rput(-0.5,0){$O$}
}
\end{pspicture}

\caption{\label{fig:transmagvec}
\small Transfert d'aimantation selon le modèle vectoriel}
\end{center}
\end{figure}

L'évolution de l'aimantation initiale des noyaux $I$ entre les
instants 0 et 3 est retracée par la figure \ref{fig:transmagvec}.
L'évolution pendant $T$ correspond à une rotation de 
$\pm J/2 \cdot 1/2J = \pm1/4$ de tour, soit un angle de $\pm \pi/2$.
La représentation vectorielle de l'état $2I_zS_z$ est constitée d'une paire
de vecteurs opposés (donc de somme nulle).
Le terme de l'état inital $\poli/2 \cdot I_z$ contribue aux populations
($p_{\al\al}$, $p_{\al\be}$, $p_{\be\al}$, $p_{\be\be}$) pour 
($\poli/4$, $\poli/4$, $-\poli/4$, $-\poli/4$) selon les équations
\ref{eqn:popisa} à \ref{eqn:popisd}.
L'état $\poli/2 \cdot 2I_zS_z$ contribue, d'après les mêmes équations
pour ($\poli/4$, -$\poli/4$, $-\poli/4$, $\poli/4$) 
aux populations des 4 états du système.
Cela revient à dire que la séquence $\pi/2_x(I)$ -- $1/2J$ -- $\pi/2_y(I)$
a inversé sélectivement les populations
des états $\al\be$ et $\be\be$, ce correspond bien a l'inversion
de la partie de l'aimantation de $I$ qui est étiquetée $\be$ en
ce qui concerne l'état du noyau $S$.
Après l'instant 3, l'intensité de la résonance de $S$ étiquetée $\al$
est proportionnelle à $p_{\al\al}$ - $p_{\al\be} = \poli/2$.
L'intensité de celle étiquetée $\be$ est $p_{\be\al}$ - $p_{\be\be} = -\poli/2$.
En résumé, l'aimantation initiale de $I$ devient à l'instant 3 la source
d'un doublet antiphase pour $S$, d'intensité totale certes nulle, mais dont
chacune des composantes est plus intense (d'un facteur $\gi/\gs$) que
les composantes en phase issues de l'aimantation initiale du noyau $S$.
L'origine du transfert d'aimantation via l'état $2I_zS_z$
peut être vue dans l'identité des membres de droite des équations
\ref{eqn:izsza} et \ref{eqn:izszb}, identité qui permet de transférer les vecteurs
opposés obtenus à l'instant 3 depuis le référentiel tournant des noyaux
$I$ vers celui des noyaux $S$.

\subsection{Le transfert INEPT}

\begin{figure}[hbt]
\begin{center}
\begin{pspicture}(0,0)(9,6.5)
% sequence I
\rput(2,5){
\psline(0,0)(7,0)
\rput(-0.5,0){RF($I$)}
\psline[linewidth=2mm]{-}(0.9,0)(0.9,1)
\rput(0.9,1.2){$\phi_1$}
\psline[linewidth=4mm]{-}(1.9,0)(1.9,1)
\rput(1.9,1.2){$\phi_2$}
\psline[linewidth=2mm]{-}(2.9,0)(2.9,1)
\rput(2.9,1.2){$\phi_3$}
\psline[linewidth=4mm]{-}(4.1,0)(4.1,1)
\rput(4.1,1.2){$\phi_4$}
\psframe(5.1,0)(6.5,0.8)
\rput(5.8,0.4){Déc.}
}
% sequence S
\rput(2,3.5){
\psline(0,0)(7,0)
\rput(-0.5,0){RF($S$)}
\psline[linewidth=4mm]{-}(1.9,0)(1.9,1)
\rput(1.9,1.2){$\phi_5$}
\psline[linewidth=2mm]{-}(3.1,0)(3.1,1)
\rput(3.1,1.2){$\phi_6$}
\psline[linewidth=4mm]{-}(4.1,0)(4.1,1)
\rput(4.1,1.2){$\phi_7$}
\rput(5.1,0){
\pscurve(0,0.5)(0.5,0.25)(1.5,0)
\pscurve(0,-0.5)(0.5,-0.25)(1.5,0)
\psline(0,0.5)(0,-0.5)
}
\rput(6,0.7){$\phi_R$}
}
% time marks
\rput(2,2.5){
\psline{->}(0,0)(7,0)
\psline[linewidth=0.25mm,linestyle=dashed]{-}(0.8,2.3)(0.8,-0.2)
\rput(0.7,-0.4){0}
\psline[linewidth=0.25mm,linestyle=dashed]{-}(1,2.3)(1,-0.2)
\rput(1.1,-0.4){1}
\rput(1.4,0.5){$\frac{T}{2}$}
\rput(2.4,0.5){$\frac{T}{2}$}
\psline[linewidth=0.25mm,linestyle=dashed]{-}(2.8,2.3)(2.8,-0.2)
\rput(2.7,-0.4){2}
\psline[linewidth=0.25mm,linestyle=dashed]{-}(3,0.8)(3,-0.2)
\rput(3,-0.4){3}
\psline[linewidth=0.25mm,linestyle=dashed]{-}(3.2,0.8)(3.2,-0.2)
\rput(3.3,-0.4){4}
\rput(3.6,0.5){$\frac{T'}{2}$}
\rput(4.6,0.5){$\frac{T'}{2}$}
\psline[linewidth=0.25mm,linestyle=dashed]{-}(5.1,0.4)(5.1,-0.2)
\rput(5.1,-0.4){5}
\psline[linewidth=0.25mm,linestyle=dashed]{-}(6,0.7)(6,-0.2)
\rput(6,-0.4){$t$}
}
% coherence order of I
\rput(2,1.5){
\psline(0,0.25)(7,0.25)
\rput(-0.5,0.3){$+1$}
\psline(0,0)(7,0)
\rput(-0.5,0){$0$}
\psline(0,-0.25)(7,-0.25)
\rput(-0.5,-0.3){$-1$}
\psline[linewidth=0.8mm]{-}(0,0)(0.8,0)(1,0.25)(1.7,0.25)(2.1,-0.25)(2.8,-0.25)(3,0)(7,0)
\psline[linewidth=0.8mm]{-}(0.8,0)(1,-0.25)(1.7,-0.25)(2.1,0.25)(2.8,0.25)(3,0)
\rput(-1.5,0){$p(I)$}
}
% coherence order of S
\rput(2,0.5){
\psline(0,0.25)(7,0.25)
\rput(-0.5,0.3){$+1$}
\psline(0,0)(7,0)
\rput(-0.5,0){$0$}
\psline(0,-0.25)(7,-0.25)
\rput(-0.5,-0.3){$-1$}
\psline[linewidth=0.8mm]{-}(0,0)(3,0)(3.2,0.25)(3.9,0.25)(4.1,-0.25)(7,-0.25)
\rput(-1.5,0){$p(S)$}
}
\end{pspicture}
\caption{\label{fig:transmaghetb}
Séquence INEPT et son chemin de transfert de cohérence.}
\end{center}
\end{figure}

L'expérience qui vient d'être décrite présente deux défauts. 
L'intensité des raies dépend de $\omsi$ selon $\cos(\omsi/2J)$
comme le montrerait un calcul plus complet.
Le découplage hétéronucléaire appliqué pendant la détection du {\FID}
annulerait l'amélioration de l'intensité du signal acquise 
grâce au transfert de polarisation : le terme $2I_zS_y$ ne fournit aucun signal 
s'il y découplage pendant l'acquisition, seule reste l'évolution de $-aS_z$.

Le remède à ces deux problèmes est le même et est apporté par l'introduction
de deux séquences d'écho de spin de durée $T$, l'une avant, 
et l'autre après l'étape de transfert d'aimantation (Figure \ref{fig:transmaghetb}).
Ces échos concernent simultanément les noyaux $I$ et $S$.
Ainsi, entre les instants 0 et 5, leurs offsets n'interviennent pas.
L'état couplé de $S$ produit par le transfert d'aimantation
est converti en un état non couplé pendant $T' = T = 1/2J$, état
qui évolue à le seule fréquence $\omss$ pendant l'acquisition
découplée des noyaux $I$.
L'aimantation initiale de $S$ est transformée en aimantation
antiphase qui reste indétectable pendant l'acquisition.

\subsection{INEPT ou effet Overhauser ?}
En RMN du \carb, l'augmentation de sensibilité obtenue par effet Overhauser est au plus
d'un facteur 1 + $\gamma$(\prot)/2$\gamma$(\carb) = 3. 
Celle apportée par la séquence INEPT est $\gamma$(\prot)/$\gamma$(\carb) = 4,
à peine légèrement supérieure, surtout s'il est tenu compte de l'effet
cumulatif des erreurs de calibration des impulsions et des effets d'offset.

Sachant que $\gamma$(\prot)/$\gamma$(\azot) = -10, l'avantage de la séquence INEPT
est clair puisque l'effet Overhauser ne peut fournir au plus qu'une amplification
des signaux d'un facteur 4 (en valeur absolue).
La réalisation pratique de spectres de RMN de l'\azot reste toutefois restreinte
aux échantillons très concentrés car l'abondance naturelle de ce noyau
n'est que 0,37 \%.

\subsection{Programme de phase}
Les phases des impulsions, de $\phi_1$ à $\phi_7$ sauf $\phi_3$,
peuvent arbitrairement être toutes prises égales à $0$, et $\phi_3$ à $\pi/2$
pour une raison exposée ci-dessus.
L'impulsion de phase $\phi_1$ crée de l'aimantation transversale ($\pm 1$ quanta)
de $I$, celle de phase $\phi_2$ inverse les ordres de cohérence et celle de
phase $\phi_3$ transforme les cohérences en populations (0 quanta).
A la suite de cela, l'impulsion de phase $\phi_6$ crée des états à $\pm 1$
quanta de $S$. 
Touefois seule la transition vers l'état où $p(S) = 1$ est matérialisé
sur la figure \ref{fig:transmaghetb}.
En effet, seul ce chemin conduit à de l'aimantation observable de $S$ ($p(S) = -1$)
après inversion des ordres de cohérence par l'impulsion de phase $\phi_7$.
Notons que les impulsions de phase $\phi_4$ et $\phi_5$ ne causent pas
de changement d'ordre de cohérence.

L'équation \ref{eqn:masterphase} relie les variations des phases des impulsions
avec celles de la phase du récepteur qui conduisent à une addition cohérente des signaux
et à l'élimination d'un certain nombre d'artefacts intrumentaux possibles, issus
des imperfections du récepteur, de la calibration imparfaite des impulsions,
des effets d'offset, ou de la durée approximative des délais.
Le chemin de transfert de cohérence pour le noyau $S$ est sélectionné par :
\begin{equation}
\Delta\phi_R = 0\Delta\phi_5 - \Delta\phi_6 + 2\Delta\phi_7
\end{equation}
Si les trois phases des impulsions sur $S$
sont augmentées simultanément de $\Delta\phi_{5,6,7}$, alors
\begin{equation}
\Delta\phi_R = \Delta\phi_{5,6,7}
\end{equation}
ce qui est nécessaire pour détecter l'aimantation transversale de $S$
formée à partir de son aimantation longitudinale, qu'elle provienne
du terme $aS_z$ ou de $I_z$ via le mécanisme de transfert.
Le programme de phase minimum pourrait donc consister à fixer
simultanément $\Delta\phi_R$, $\Delta\phi_5$, $\Delta\phi_6$ et $\Delta\phi_7$
à $\Delta\phi_{5,6,7} = \pi/2$, bien que $\Delta\phi_{5,6,7} = \pi$ pourrait
suffire, sachant que les pics issus du défaut de quadrature ont une
intensité qui les rend généralement indétectables dans le bruit du spectre.
De manière équivalente, il suffit de cycler identiquement les phases
de $\phi_R$, $\phi_6$ et $\phi_7$.
Le cyclage des phase $\phi_5$ et $\phi_7$ est susceptible de compenser les
défauts de ces impulsions d'angle $\pi$, sans que cela fasse 
ici l'objet d'une démonstration.

\renewcommand{\baselinestretch}{1}
\normalsize
\begin{table}[hbt]
\begin{center}
\begin{tabular}{ccccccccccccccccc}
pas          &  1 &  2 &  3 &  4 &  5 &  6 &  7 &  8 \\
             &  9 & 10 & 11 & 12 & 13 & 14 & 15 & 16 \\
\hline
$\phi_1$ (4) &  0 &  0 &  0 &  0 &  0 &  0 &  0 &  0 \\
             &  2 &  2 &  2 &  2 &  2 &  2 &  2 &  2 \\
$\phi_2$ (4) &  0 &  2 \\
$\phi_3$ (4) &  1 &  1 &  3 &  3 \\
$\phi_4$ (4) &  0 &  2 \\
$\phi_5$ (4) &  0 &  2 \\
$\phi_6$ (4) &  0 &  0 &  0 &  0 &  1 &  1 &  1 &  1 \\
             &  2 &  2 &  2 &  2 &  3 &  3 &  3 &  3 \\
$\phi_7$ (4) &  0 &  2 &  0 &  2 &  1 &  3 &  1 &  3 \\
$\phi_R$ (4) &  0 &  0 &  2 &  2 &  1 &  1 &  3 &  3 \\
\hline
\end{tabular}
\caption{\label{tab:inept}
Programme de phase de l'expérience INEPT}
\end{center}
\end{table}
\renewcommand{\baselinestretch}{1.5}
\normalsize

En ce qui concerne le noyau $I$, il serait par exemple possible
de cycler $\phi_1$ indépendamment des autres phases.
Pour garder les deux chemins correspondants à $\Delta p = +1$ et
$\Delta p = -1$, soit $\Delta(\Delta p) = 2$, il faut choisir
$\Delta \phi_1 = \pi$, comme indiqué par l'équation \ref{eqn:selectionphase}.
Il en est de même pour $\phi_3$.
Dans les deux cas $\phi_R$ reste inchangée.
L'incrément de phase $\Delta \phi_2 = \pi/2$, et à 
plus forte raison $\Delta \phi_2 = \pi$, préserve les deux chemins.
où $\Delta p = +2$ et $\Delta p = -2$.
Le fait d'imposer aussi $\phi_3 - \phi_1 = \pm \pi/2$ ne relève pas 
de la théorie du programme de phase.
Il n'est en effet pas suffisant de sélectionner un ou des chemins,
il faut aussi que tous les transferts soient associés à un coefficient
de transfert non nul pour qu'il signal soit détecté.

Parmi les choix possibles, le programme de phase de la table \ref{tab:inept}
satisfait aux nécessités énoncées ci-dessus.
L'alternance des phases $\phi_2$, $\phi_4$, $\phi_5$ et $\phi_7$ des impulsions
d'angle $\pi$ ne cause aucun changement de $\phi_R$ (pas pairs et impairs).
L'inversion de $\phi_3$ entraîne celle de $\phi_R$ (pas 1 et 3, 2 et 4, etc...).
L'augmentation simultanée de $\phi_6$ et $\phi_7$ de $\pi/2$ nécessite
une augmentation identique de $\phi_R$ (pas 1 et 5, 2 et 6, etc...).
Finalement, l'inversion de $\phi_1$ cause celle de $\phi_R$ (pas 1 et 9, 2 et 10, etc...).
Le cyclage n'est pas total sur l'ensemble de toutes les impulsions pour
que le programme de phase reste de dimension raisonnable
tout en éliminant les causes principales d'artefacts.

\subsection{Edition des spectres par la séquence INEPT}
Un atome $S$ lié à aucun atome $I$ (un carbone quaternaire, par exemple)
ne fournit aucun signal puisqu'une inversion de $\phi_1$ ou de $\phi_3$,
à laquelle les noyaux $S$ est insensible, s'accompagne de l'inversion de
$\phi_R$ et donc de la disparition du signal par soustraction.
L'analyse du comportement d'un système $I_2S$ et $I_3S$ fait apparaître que
si $T = T' = 1/2J(IS)$, la séquence INEPT produit un signal nul.
Si $\pi J T' = \alpha$, l'intensité des signaux 
issus des groupes $IS$, $I_2S$ et $I_3S$ dépend de $\alpha$ selon une
loi qui leur est spécifique.
Des combinaisons linéaires des spectres obtenus permettent de fabriquer
des sous-spectres dans lesquelles n'apparaissent que les signaux
des noyaux $S$ liés à 1, 2 ou 3 de noyaux $I$.
L'opération ainsi effectuée s'appelle "édition des spectres".

\begin{figure}[hbt]
\begin{center}
\begin{pspicture}(0,0)(7,4.5)
% sequence I
\rput(1,3){
\psline(0,0)(6,0)
\rput(-0.5,0){RF($I$)}
\psline[linewidth=2mm]{-}(0.9,0)(0.9,1)
\rput(0.9,1.2){$x$}
\psline[linewidth=2mm]{-}(2.9,0)(2.9,1)
\rput(2.9,1.2){$y$}
\psframe(4.1,0)(5.5,0.8)
\rput(4.8,0.4){Déc.}
}
% sequence S
\rput(1,1.5){
\psline(0,0)(6,0)
\rput(-0.5,0){RF($S$)}
\psline[linewidth=2mm]{-}(3.1,0)(3.1,1)
\rput(3.1,1.2){$-x$}
\rput(4.1,0){
\pscurve(0,0.5)(0.5,0.25)(1.5,0)
\pscurve(0,-0.5)(0.5,-0.25)(1.5,0)
\psline(0,0.5)(0,-0.5)
}
\rput(5,0.7){$\phi_R$}
}
% time marks
\rput(1,0.5){
\psline{->}(0,0)(6,0)
\psline[linewidth=0.25mm,linestyle=dashed]{-}(0.8,2.3)(0.8,-0.2)
\rput(0.7,-0.4){0}
\psline[linewidth=0.25mm,linestyle=dashed]{-}(1,2.3)(1,-0.2)
\rput(1.1,-0.4){1}
\rput(1.9,0.5){$T$}
\psline[linewidth=0.25mm,linestyle=dashed]{-}(2.8,2.3)(2.8,-0.2)
\rput(2.7,-0.4){2}
\psline[linewidth=0.25mm,linestyle=dashed]{-}(3,0.8)(3,-0.2)
\rput(3,-0.4){3}
\psline[linewidth=0.25mm,linestyle=dashed]{-}(3.2,0.8)(3.2,-0.2)
\rput(3.3,-0.4){4}
\rput(3.65,0.5){$T'$}
\psline[linewidth=0.25mm,linestyle=dashed]{-}(4.1,0.4)(4.1,-0.2)
\rput(4.1,-0.4){5}
\psline[linewidth=0.25mm,linestyle=dashed]{-}(5,0.7)(5,-0.2)
\rput(5,-0.4){$t$}
}
\end{pspicture}
\caption{\label{fig:ineptsimple}
Séquence INEPT simplifiée pour l'analyse de l'édition spectrale.}
\end{center}
\end{figure}

Pour simplifier le travail d'analyse, la séquence de la figure \ref{fig:ineptsimple}
sera considérée, en imposant de plus $\omss = 0$ et $\omsi = 0$
puisque les échos de spin de la figure \ref{fig:transmaghetb} ont été supprimés.
Seule l'aimantation initiale des noyaux $I$ sera prise en compte sachant que celle
des noyaux $S$ produit un signal qui est éliminé par le programme de phases.

\subsubsection{Système $IS$}
A l'instant 4 l'état du système est décrit par l'équation \ref{eqn:fintransmag}.
Ainsi :
\begin{equation}
\sigma_5 = - 2I_zS_y \ca + S_x \sa
\end{equation}
sachant que par commodité d'écriture la phase de l'impulsion sur $S$ a été inversée.
La partie $\sigma_5'$ de $\sigma_5$ qui contribue au signal mesurable pendant
le découplage des noyaux $I$ est 
\begin{equation}
\sigma_5' (IS) = \sa S_x
\end{equation}
et qui fournit un signal d'intensité maximale quand $\alpha = \pi/2$.

\subsubsection{Système $I_2S$}
Ce système sera traité comme un système $II'S$ où $\Omega_{I'} = 0$,
$J_{II'} = 0$ et $J_{I'S} = J_{IS} = J$.
De l'état initial
\begin{equation}
\sigma_0 = I_z + I'_z + aS_z
\end{equation}
seule l'évolution du premier terme sera analysée, sachant que celle du second terme
est identique par symétrie entre $I$ et $I'$ et que celle du troisième
ne contribue pas au signal.
Comme pour le système $IS$, $\sigma_4 = -2I_zS_y$.
Pendant $T'$, ce terme évolue sous l'action de $\alpha 2I_zS_z$ et de $\alpha 2I'_zS_z$
puisqu'aucun de ces deux opérateur ne commute avec $\sigma_4$.
Pour que de l'aimantation non couplée de $S$ soit produite à l'instant 5, il faut
ne considérer que la production d'un terme $S_x$ par action de $\alpha 2I_zS_z$ sur
$\sigma_4$ puis la conservation de ce terme $S_x$ par action de $\alpha 2I'_zS_z$:
\begin{equation}
-2I_zS_y \flham{\alpha 2I_zS_z} \sa S_x + \cdots \flham{\alpha 2I'_zS_z} \sa\ca S_x + \cdots
\end{equation}
L'addition des contributions des noyaux $I$ et $I'$ aboutit à
\begin{equation}
\sigma_5' (I_2S) = 2\sa\ca S_x
\end{equation}
qui est bien nulle si $T' = T$ puisque dans ce cas $\alpha = \pi/2$ et donc $\sa = 0$.

\subsubsection{Système $I_3S$}
Ce système sera traité comme un système $II'I"S$ où $\Omega_{I'} = \Omega_{I"} = 0$,
$J_{II'} = J_{II"} = J_{I'I"} = 0$ et $J_{I"S} = J_{I'S} = J_{IS} = J$.
La démarche exposée pour un système $I_2S$ s'étend sans difficulté au système $I_3S$.
Ainsi :
\begin{equation}
\sigma_5' (I_3S) = 3\cos^2\alpha\sa S_x
\end{equation}

Si $\alpha = \pi/4$ alors les signaux des systèmes $IS$, $I_2S$ et $I_3S$
fournissent des pics spectraux d'intensités toutes de même signe, si
$\alpha = \pi/2$, seuls les pics des systèmes $IS$ sont visibles, et enfin
si $\alpha = 3\pi/4$ les intensités des pics des groupes $I_2S$ sont de signe opposé
à ceux des groupes $IS$ et $I_3S$.
L'enregistrement de trois spectres avec $\alpha = \pi/4$, $\pi/2$ et $3\pi/4$
permet donc de déterminer le nombre de noyaux $I$ attachés au noyau $S$.
Si ce nombre est nul, l'expérience impulsion -- détection permet de visualiser
les signaux correspondants.
L'expérience INEPT n'est généralement pas utilisée pour ses possibilités
d'édition spectrale car toutes les valeurs de $J$ ne sont pas identiques
dans une molécule donnée.
La dispersion des valeurs de $J$ conduit à des spectres difficiles à phaser
et les méthodes utilisées sont soit l'enregistrement de spectres $J$-modulés
ou de spectres DEPT.
Toutefois, les considérations développées ci-dessus 
ne sont pas inutiles car elles montrent que l'aimantation
transversale des noyaux $S$ évolue de manières distinctes selon le
nombre de noyaux $I$ attachés.
De plus, le transfert d'aimantation INEPT est un élément présent dans de nombreuses
séquences impulsionnelles en RMN 2D hétéronucléaire.

\section{Spectres DEPT}
Les séquences INEPT et DEPT sont l'une comme l'autre utilisables
pour éditer des sous-spectres de noyaux $S$ en fonction du nombre de noyaux $I$ voisins. 
La séquence DEPT (Distortionless Enhancement by Polarisation Transfer)
présente l'avantage de n'être que peu sensible à la dispersion
des valeurs des constantes de couplage $J_{IS}$.
Elle est aussi basée sur le transfert d'aimantation des noyaux $I$ vers les noyaux $S$.
L'analyse de son fonctionnement sera effectuée d'abord sur une version simplifiée
pour laquelle il est nécessaire de considérer $\omsi = \omss = 0$
(Fig. \ref{fig:deptsimple}).

\begin{figure}[hbt]
\begin{center}
\begin{pspicture}(0,0)(10,4.5)
% sequence I
\rput(1,3){
\psline(0,0)(9,0)
\rput(-0.5,0){RF($I$)}
\psline[linewidth=2mm]{-}(1,0)(1,1)
\rput(1,1.2){$\pi/2_x$}
\psline[linewidth=2mm]{-}(5,0)(5,1)
\rput(5,1.2){$\alpha_y$}
\psframe(6.9,0)(8.5,0.8)
\rput(7.7,0.4){Déc.}
}
% sequence S
\rput(1,1.5){
\psline(0,0)(9,0)
\rput(-0.5,0){RF($S$)}
\psline[linewidth=2mm]{-}(3,0)(3,1)
\rput(3,1.2){$\pi/2_x$}
\rput(6.9,0){
\pscurve(0,0.5)(0.5,0.25)(1.5,0)
\pscurve(0,-0.5)(0.5,-0.25)(1.5,0)
\psline(0,0.5)(0,-0.5)
}
\rput(8,0.7){$\phi_R$}
}
% time marks
\rput(1,0.5){
\psline{->}(0,0)(9,0)
\psline[linewidth=0.25mm,linestyle=dashed]{-}(0.9,2.3)(0.9,-0.2)
\rput(0.8,-0.4){0}
\psline[linewidth=0.25mm,linestyle=dashed]{-}(1.1,2.3)(1.1,-0.2)
\rput(1.2,-0.4){1}
\rput(2,0.5){$1/2J$}
\psline[linewidth=0.25mm,linestyle=dashed]{-}(2.9,0.8)(2.9,-0.2)
\rput(2.8,-0.4){2}
\psline[linewidth=0.25mm,linestyle=dashed]{-}(3.1,0.8)(3.1,-0.2)
\rput(3.2,-0.4){3}
\rput(4,0.5){$1/2J$}
\psline[linewidth=0.25mm,linestyle=dashed]{-}(4.9,2.3)(4.9,-0.2)
\rput(4.8,-0.4){4}
\psline[linewidth=0.25mm,linestyle=dashed]{-}(5.1,2.3)(5.1,-0.2)
\rput(5.2,-0.4){5}
\rput(6,0.5){$1/2J$}
\psline[linewidth=0.25mm,linestyle=dashed]{-}(6.9,0.4)(6.9,-0.2)
\rput(6.9,-0.4){6}
\psline[linewidth=0.25mm,linestyle=dashed]{-}(8,0.7)(8,-0.2)
\rput(8,-0.4){$t$}
}
\end{pspicture}
\caption{\label{fig:deptsimple}
Séquence DEPT simplifiée}
\end{center}
\end{figure}

Quel que soit le nombre de noyaux $I$ liés au noyau $S$, l'aimantation
de chaque noyau $I$ est produit un état à 0 et $\pm2$ quanta à l'instant 3 :
\begin{eqnarray}
\sigma_0 & = & I_z \\
\sigma_1 & = & -I_y \\
\sigma_2 & = & 2I_xS_z \\
\sigma_3 & = & -2I_xS_y
\end{eqnarray}

L'effet du second délai $1/2J$ est de produire des états à 0 et $\pm2$ quanta
couplés des noyaux $I'$ et $I"$ quand ceux-ci existent.
Notons que $\sigma_3$ commute avec $2I_zS_z$.
Ainsi :

\begin{eqnarray}
\mbox{Système} & & \sigma_4 \nonumber\\
IS & : & -2I_xS_y \\
I_2S & : & +4I_xI'_zS_x \\
I_3S & : & +8I_xI'_zI''_zS_y
\end{eqnarray}

Il faut que la seconde impulsion sur les noyaux $I$ soit de phase $y$ pour
que $\sigma_5$ et donc $\sigma_6$ soit un état à zéro quanta de $I$.
L'angle de cette impulsion pouvant être de valeur arbitraire, seule la partie de $\sigma_5$
qui est à 0 quanta pour $I$ sera conservée dans la suite du calcul.
La transformation de $I_x$ en $I_z$ est liée à un coefficient multiplicatif $\sa$,
alors que la préservation de $I'_z$ ou de $I''_z$ est liée à $\ca$ : 
\begin{eqnarray}
\mbox{Système} & & \sigma_5 \nonumber\\
IS & : & + \sa \cdot 2I_zS_y + \cdots \\
I_2S & : & - \sa\ca \cdot 4I_zI'_zS_x + \cdots \\
I_3S & : & - \sa\cos^2\alpha \cdot 8I_zI'_zI''_zS_y + \cdots
\end{eqnarray}
Pendant le troisième délai, l'action du ou des couplages conduit à l'état non
couplé $S_x$ modulé par la valeur de l'angle $\alpha$ :
\begin{eqnarray}
\mbox{Système} & & \sigma_6 \nonumber\\
IS & : & - \sa \cdot S_x + \cdots \\
I_2S & : & + \sa\ca \cdot S_x + \cdots \\
I_3S & : & - \sa\cos^2\alpha \cdot S_x + \cdots
\end{eqnarray}

\begin{figure}[hbt]
\begin{center}
\begin{pspicture}(0,0)(11,6.5)
% sequence DEPT
\rput(1,2){
% sequence I
\rput(1,3){
\psline(0,0)(9,0)
\rput(-0.5,0){RF($I$)}
\psline[linewidth=2mm]{-}(1,0)(1,1)
\rput(1,1.2){$\pi/2_x$}
\psline[linewidth=4mm]{-}(3,0)(3,1)
\rput(3,1.2){$\pi$}
\psline[linewidth=2mm]{-}(5,0)(5,1)
\rput(5,1.2){$\alpha_y$}
\psframe(6.9,0)(8.5,0.8)
\rput(7.7,0.4){Déc.}
}
% sequence S
\rput(1,1.5){
\psline(0,0)(9,0)
\rput(-0.5,0){RF($S$)}
\psline[linewidth=2mm]{-}(3,0)(3,1)
\rput(3,1.2){$\pi/2_x$}
\psline[linewidth=4mm]{-}(5,0)(5,1)
\rput(5,1.2){$\pi$}
\rput(6.9,0){
\pscurve(0,0.5)(0.5,0.25)(1.5,0)
\pscurve(0,-0.5)(0.5,-0.25)(1.5,0)
\psline(0,0.5)(0,-0.5)
}
\rput(8,0.7){$\phi_R$}
}
% time marks
\rput(1,0.5){
\psline{->}(0,0)(9,0)
\psline[linewidth=0.25mm,linestyle=dashed]{-}(0.9,2.3)(0.9,-0.2)
\rput(0.8,-0.4){0}
\psline[linewidth=0.25mm,linestyle=dashed]{-}(1.1,2.3)(1.1,-0.2)
\rput(1.2,-0.4){1}
\rput(2,0.5){$1/2J$}
\psline[linewidth=0.25mm,linestyle=dashed]{-}(2.9,0.8)(2.9,-0.2)
\rput(2.8,-0.4){2}
\psline[linewidth=0.25mm,linestyle=dashed]{-}(3.1,0.8)(3.1,-0.2)
\rput(3.2,-0.4){3}
\rput(4,0.5){$1/2J$}
\psline[linewidth=0.25mm,linestyle=dashed]{-}(4.9,0.8)(4.9,-0.2)
\rput(4.8,-0.4){4}
\psline[linewidth=0.25mm,linestyle=dashed]{-}(5.1,0.8)(5.1,-0.2)
\rput(5.2,-0.4){5}
\rput(6,0.5){$1/2J$}
\psline[linewidth=0.25mm,linestyle=dashed]{-}(6.9,0.4)(6.9,-0.2)
\rput(6.9,-0.4){6}
\psline[linewidth=0.25mm,linestyle=dashed]{-}(8,0.7)(8,-0.2)
\rput(8,-0.4){$t$}
}
}
% p(I)
\rput(2,1.5){
\psline(0,0.25)(9,0.25)
\rput(-0.5,0.3){$+1$}
\psline(0,0)(9,0)
\rput(-0.5,0){$0$}
\psline(0,-0.25)(9,-0.25)
\rput(-0.5,-0.3){$-1$}
\psline[linewidth=0.8mm]{-}(0,0)(0.9,0)(1.1,0.25)(2.8,0.25)(3.2,-0.25)(4.9,-0.25)(5.1,0)(9,0)
\psline[linewidth=0.8mm]{-}(0.9,0)(1.1,-0.25)(2.8,-0.25)(3.2,0.25)(4.9,0.25)(5.1,0)
\rput(-1.5,0){$p(I)$}
}
% p(S)
\rput(2,0.5){
\psline(0,0.25)(9,0.25)
\rput(-0.5,0.3){$+1$}
\psline(0,0)(9,0)
\rput(-0.5,0){$0$}
\psline(0,-0.25)(9,-0.25)
\rput(-0.5,-0.3){$-1$}
\psline[linewidth=0.8mm]{-}(0,0)(2.9,0)(3.1,0.25)(4.8,0.25)(5.2,-0.25)(9,-0.25)
\rput(-1.5,0){$p(S)$}
}
\end{pspicture}
\caption{\label{fig:dept}
Séquence DEPT complète}
\end{center}
\end{figure}

La séquence complète (Fig. \ref{fig:dept}) se distingue de la séquence simplifiée
par la présence d'impulsions de refocalisation situées au milieu
des périodes d'évolution de l'aimantation transversale des noyaux
$I$ (entre les instants 1 et 4) et $S$ (instants 3 et 6).
Le résultat produit par cette séquence d'impulsions est alors indépendant
des valeurs de $\omsi$ et $\omss$.
Le calcul de $\sigma_6$ à partir de $\sigma_5$ fait aussi apparaître une inversion de
signe pour les systèmes $IS$ et $I_3S$.
Dans ces deux cas, $\sigma_5$ contient la matrice de base $S_y$ dont le signe
est inversé par application de l'opérateur $\pi S_x$.
En tenant compte du fait que chaque noyau de type $I$ contribue au signal $s(t)$ et
donc au spectre mesuré, l'intensité relative du pic produit par le noyau $S$ 
d'un système $I_nS$ est proportionnelle à $n\sa\cos^{n-1}\alpha$.

Pratiquement, il faut enregistrer trois spectres $S(\pi/4)$, $S(\pi/2)$ et $S(3\pi/4)$.
Si $S_1$, $S_2$ et $S_3$ désignent les spectres issus des systèmes
$IS$, $I_2S$ et $I_3S$ de la substance étudiée, considérés séparément et affectés des
coefficients 1, 2 et 3, alors
\begin{equation}
\label{eqn:mixdept}
S(\alpha) = \sum_{n=1}^3\sa\cos^{n-1}\alpha \cdot S_n = \sum_{n=1}^3 c_n(\alpha) S_n
\end{equation}
Les coefficients $c_n(\alpha)$ sont donnés dans la table \ref{tab:mixdept}

\begin{figure}[hbt]
\begin{center}
\begin{tabular}{c|ccc}
$c_n(\alpha)$  & 1 & 2 & 3 \\
\hline
$\pi/4$ & $\sqrt{2}/2$ & 1/2 & $\sqrt{2}/4$ \\
$\pi/2$ & 1 & 0 & 0 \\
$3\pi/4$ & $\sqrt{2}/2$ & -1/2 & $\sqrt{2}/4$
\end{tabular}
\caption{\label{tab:mixdept}
Intensités relatives $c_n(\alpha)$ des pics fournis par la séquence DEPT}
\end{center}
\end{figure}

L'édition spectrale consiste à inverser l'équation \ref{eqn:mixdept} par application
de l'équation \ref{eqn:unmixdept}:
\begin{equation}
\label{eqn:unmixdept}
S_n = \sum_{n=1}^3 d_n(\alpha) S(\alpha)
\end{equation}
où les coefficients $d_n(\alpha)$ sont fournis par la table \ref{tab:unmixdept}.

\begin{figure}[hbt]
\begin{center}
\begin{tabular}{c|ccc}
$d_n(\alpha)$  & $\pi/4$ & $\pi/2$ & $3\pi/4$ \\
\hline
1 & 0 & 1 & 0 \\
2 & 1 & 0 & -1 \\
3 & $\sqrt{2}$ & -2 & $\sqrt{2}$
\end{tabular}
\caption{\label{tab:unmixdept}
Édition spectrale par combinaison de spectres DEPT}
\end{center}
\end{figure}

Dans les conditions réelles d'utilisation, les impulsions peuvent ne pas être
parfaitement calibrées.
Les coeffients $c_n(\alpha)$ sont alors différents de ceux donnés dans la table
\ref{tab:mixdept}.
L'édition spectrale requiert alors un ajustement manuel des coefficients $d_n(\alpha)$
de la table \ref{tab:unmixdept}.

Le spectre $S(3\pi/4)$ aussi appelé spectre DEPT-135 ($3\pi/4 = 135^{\circ}$) contient
souvent l'information recherchée, pour peu que les valeurs des déplacements
chimiques des noyaux $S$ ou qu'un spectre 2D permette de distinguer les systèmes
$IS$ des systèmes $I_3S$.
Il n'est pas alors nécessaire de recourir à la procédure DEPT dans son ensemble.

\section{Spectre $J$-modulé}
La séquence d'impulsions $J$-modulée \ref{fig:jmod}
permet de discriminer les systèmes de spins
$I_nS$ selon la parité de $n$, $n=0$ inclus.
Elle donne la même information que le séquence DEPT-135, avec l'avantage
supplémentaire de faire apparaître les signaux des noyaux $S$ isolés.

\begin{figure}[hbt]
\begin{center}
\begin{pspicture}(0,0)(10,4.5)
% sequence I
\rput(1,3){
\psline(0,0)(9,0)
\rput(-0.5,0){RF($I$)}
\psframe(1,0)(2.9,0.8)
\rput(2,0.4){Sat.}
\psframe(5.2,0)(8.5,0.8)
\rput(6.9,0.4){Déc.}
}
% sequence S
\rput(1,1.5){
\psline(0,0)(9,0)
\rput(-0.5,0){RF($S$)}
\psline[linewidth=2mm]{-}(3,0)(3,1)
\rput(3,1.2){$\pi/2_x$}
\psline[linewidth=4mm]{-}(5,0)(5,1)
\rput(5,1.2){$\pi_x$}
\rput(6.9,0){
\pscurve(0,0.5)(0.5,0.25)(1.5,0)
\pscurve(0,-0.5)(0.5,-0.25)(1.5,0)
\psline(0,0.5)(0,-0.5)
}
\rput(8,0.7){$\phi_R$}
}
% time marks
\rput(1,0.5){
\psline{->}(0,0)(9,0)
\psline[linewidth=0.25mm,linestyle=dashed]{-}(2.9,0.8)(2.9,-0.2)
\rput(2.8,-0.4){0}
\psline[linewidth=0.25mm,linestyle=dashed]{-}(3.1,0.8)(3.1,-0.2)
\rput(3.2,-0.4){1}
\rput(4,0.5){$1/J$}
\psline[linewidth=0.25mm,linestyle=dashed]{-}(4.8,0.8)(4.8,-0.2)
\rput(4.8,-0.4){2}
\psline[linewidth=0.25mm,linestyle=dashed]{-}(5.2,0.8)(5.2,-0.2)
\rput(5.2,-0.4){3}
\rput(6,0.5){$1/J$}
\psline[linewidth=0.25mm,linestyle=dashed]{-}(6.9,0.4)(6.9,-0.2)
\rput(6.9,-0.4){4}
\psline[linewidth=0.25mm,linestyle=dashed]{-}(8,0.7)(8,-0.2)
\rput(8,-0.4){$t$}
}
\end{pspicture}
\caption{\label{fig:jmod}
Séquence $J$-modulée}
\end{center}
\end{figure}

Les noyaux $S$ dipolairement couplés aux noyaux $I$ voient leur
aimantation d'équilibre augmentée par l'effet Overhauser dû
à la saturation de l'aimantation des noyaux $I$.

Un noyau $S$ non couplé scalairement va subir un écho de spin d'hamiltonien effectif nul :
\begin{eqnarray}
\sigma_0 & = & S_z \\
\sigma_1 & = & -S_y \\
\sigma_4 & = & S_y
\end{eqnarray}
Il n'est pas possible d'appliquer directement la règle de l'hamiltonien moyen
si le noyau $S$ est couplé à un ou plusieurs noyaux $I$ puisque l'écho
n'est pas de structure symétrique par rapport à l'impulsion de refocalisation.
En prenant l'exemple d'un système $I_3S$, la succession des opérateurs qui
agissent entre les instants 1 et 4 est :
\begin{equation}
\flham{\pi \cdot 2I_zS_z} \quad \flham{\pi \cdot 2I'_zS_z} 
\quad \flham{\pi \cdot 2I''_zS_z} 
\quad \flham{\omss/J \cdot S_z} \quad \flham{\pi \cdot S_x} 
\quad \flham{\omss/J \cdot S_z}
\end{equation}
Chacun des trois premiers opérateurs changent le signe de $\sigma_1$.
Les trois derniers opérateurs constituent un écho de spin comme si $S$ était isolé.
En conséquence, tout se passe comme si les opérateurs
\begin{equation}
\flham{\pi \cdot 2I_zS_z} \quad \flham{\pi \cdot 2I'_zS_z} \quad \flham{\pi \cdot 2I''_zS_z} 
\quad \flham{\pi \cdot S_x}
\end{equation}
agissent sur $\sigma_1 = -S_y$.

La généralisation aux systèmes $IS$ et $I_2S$ est immédiate et conduit à :
\begin{eqnarray}
\mbox{Système} & & \sigma_4 \nonumber\\
S & : & + S_y \\
IS & : & - S_y \\
I_2S & : & + S_y \\
I_3S & : & - S_y
\end{eqnarray}
Les signes des pics spectraux de deux noyaux $S$ sont opposés lorsque les parités des nombres
de noyaux $I$ avec qui ils couplent sont différentes.

Le programme de phase de l'expérience $J$-modulée est celui d'une séquence
d'écho de spin, comme décrit dans le tableau 
\ref{tab:impechodetec}, page \pageref{tab:impechodetec}.

