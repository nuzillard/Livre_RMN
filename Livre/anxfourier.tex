\chapter{Transformation de Fourier}
\label{chap:fourier}

Bien que d'innombrables ouvrages existent sur le sujet, cette annexe,
sans prétention de rigueur mathématique aucune, est destinée à montrer
que les équations qui définissent la TF sont accessibles sans trop de
développements théoriques.

\section{Série de Fourier}
Le concept de d'analyse en séries de Fourier ne s'applique qu'aux fonctions
périodiques.
Son étude constitue cependant le préliminaire à celle de la TF.
Une fonction $x(t)$ de la variable réelle $t$ et à valeurs réelles
est périodique de période $T$ si
\begin{equation}
x(t+kT) = x(t) \qavecq k \in \mathbb{Z}
\end{equation}
En conséquence, la connaissance de $x$ pour $t$ dans l'intervalle $[t_0, t_0+T[$
suffit pour définir $x$ partout.
La relation
\begin{equation}
\omz = 2\pi\nu_0 = \frac{2\pi}{T}
\end{equation}
définit la fréquence $\nu_0$ et la pulsation $\omz$ de la fonction $x(t)$.

Le développement de $x(t)$ en série de Fourier s'écrit :
\begin{equation}
\label{eqn:serfourdef}
x(t) = \frac{a_0}{2}+\sum_{n=1}^{\infty}
\left[a_n \cos(n\omz t) + b_n \sin(n\omz t) \right]
\end{equation}

Le fait que le coefficient $a_0$ soit divisé par 2 dans le premier terme
est une commodité qui prendra son sens un peu plus tard.
Ainsi, $x(t)$ est la somme d'une constante $a_0/2$ et d'une somme de fonctions
cosinus et sinus de pulsations $\omz$, $2\omz$, $3\omz$, $\ldots$.
Ces dernières étant des fonctions dont la valeur moyenne sur une période est nulle,
il en ressort que $a_0/2$ représente la valeur moyenne de $x(t)$.
La fonction $a_1 \cos(\omz t) + b_1 \sin(\omz t)$ ($n=1$) est de même
pulsation que $x(t)$ et s'appelle la composante fondamentale de $x(t)$.
Toute fonction sinusoïdale $A\cos(\omz t+\phi)$ d'amplitude $A$, de pulsation
$\omz$ et de phase à l'origine $\phi$ est susceptible de se mettre sous cette forme,
avec $a_1 = A \cos\phi$ et $b_1 = -\sin\phi$.
Les fonctions $a_n \cos(n\omz t) + b_n \sin(n\omz t)$ avec $n>1$ sont appelées harmoniques
d'ordre $n$.
L'ensemble des valeurs $a_0, a_1, \ldots, a_n, \ldots, b_1, \ldots, b_n, \ldots$
définissent entièrement la fonction $x(t)$ et sont les coefficient du
développement de $s(t)$ en série de Fourier.
Une fonction $x(t)$ étant donnée, le problème est savoir calculer ces coefficients.

Le calcul des coefficients de Fourier repose sur les résultats suivants :
\begin{eqnarray}
\label{eqn:intcosmcosn}
\intfou \cos(n\omz t)\cos(m\omz t) dt & = & 0 \quad\mbox{si}\quad n \ne m \\
\label{eqn:intsinmsinn}
\intfou \sin(n\omz t)\sin(m\omz t) dt & = & 0 \quad\mbox{si}\quad n \ne m \\
\label{eqn:intcossin}
\intfou \cos(n\omz t)\sin(m\omz t) dt & = & 0
\end{eqnarray}
et
\begin{equation}
\label{eqn:intcossqr}
\intfou \cos^2(n\omz t) dt = \intfou \sin^2(n\omz t) dt = \frac{T}{2}
\end{equation}

Ces résultats s'obtiennent aisément en considérant les identités trigonométriques
\begin{eqnarray}
\cos a \cos b & = & \frac{1}{2}\left[ \cos(a-b) + \cos(a+b) \right] \\
\sin a \sin b & = & \frac{1}{2}\left[ \cos(a-b) - \cos(a+b) \right] \\
\sin a \cos b & = & \frac{1}{2}\left[ \sin(a+b) - \sin(a-b) \right]
\end{eqnarray}
avec $a = n\omz t$ et $b = m\omz t$.
Sachant que $n > 0$ et $m > 0$, $a-b$ et $a+b$ sont non nuls lorsque $a \ne b$.
Il en résulte que le membre de gauche des relations \ref{eqn:intcosmcosn}
et \ref{eqn:intsinmsinn} est nul car les fonctions intégrées sont de moyenne nulle
sur un nombre entier de périodes.
Même si $a = b$ ($n = m$), la fonction intégrée dans la relation \ref{eqn:intcossin}
conduit aussi à une valeur nulle.

La valeur de $a_0$ s'obtient en intégrant les deux membres de l'équation
\ref{eqn:serfourdef} sur une période :
\begin{equation}
\intfou x(t) dt = \intfou \frac{a_0}{2} dt = \frac{a_0 T}{2} 
\end{equation}
d'où
\begin{equation}
\label{eqn:coeffazero}
a_0 = \frac{2}{T}\intfou s(t) dt 
\end{equation}
Chaque coefficient $a_n$ est obtenu en multipliant les deux membres de l'équation
\ref{eqn:serfourdef} par $\cos(n \omz t)$ et en les intégrant sur une période :
\begin{equation}
\intfou x(t) \cos(n \omz t)dt = a_n\intfou \cos^2(n\omz t) dt = \frac{a_n T}{2}
\end{equation}
d'où
\begin{equation}
\label{eqn:coeffan}
a_n = \frac{2}{T}\intfou x(t) \cos(n\omz t) dt
\end{equation}
qui reste valable quand $n=0$ (à cause du facteur 1/2 appliqué à $a_0$ dans
l'équation \ref{eqn:serfourdef}.
De même,
\begin{equation}
\label{eqn:coeffbn}
b_n = \frac{2}{T}\intfou x(t) \sin(n\omz t) dt
\end{equation}
qui reste valable si $n=0$, en définissant $b_0 = 0$.

Les coefficients $a_n$ et $b_n$ peuvent être définis pour $n<0$ 
à partir des relations \ref{eqn:coeffan} et \ref{eqn:coeffbn} :
\begin{equation}
a_{-n} = a_n \qetq b_{-n} = -b_n
\end{equation}

Le remplacement dans l'équation \ref{eqn:serfourdef}
des fonctions sinus et cosinus par les fonctions
exponentielles complexes équivalentes fournit :
\begin{eqnarray}
x(t) & = & \frac{a_0}{2} + 
\sum_{n=1}^{\infty} a_n \frac{\exp(in\omz t)+\exp(-in\omz t)}{2}
\nonumber\\
& & \quad \quad \quad 
+\sum_{n=1}^{\infty} b_n \frac{\exp(in\omz t)+\exp(-in\omz t)}{2i} \\
x(t) & = & \frac{a_0}{2} + 
\sum_{n=1}^{\infty} \frac{a_n-ib_n}{2}\exp(in\omz t)
\nonumber\\
& & \quad \quad \quad 
+\sum_{n=1}^{\infty} \frac{a_n+ib_n}{2}\exp(-in\omz t) \\
x(t) & = & \frac{a_0}{2} + 
\sum_{n=1}^{\infty} \frac{a_n-ib_n}{2}\exp(in\omz t)
\nonumber\\
& & \quad \quad \quad 
+\sum_{n=1}^{\infty} \frac{a_{-n}-ib_{-n}}{2}\exp(i(-n)\omz t) \\
x(t) & = & \sum_{n=-\infty}^{\infty} \frac{a_n-ib_n}{2}\exp(in\omz t)
\end{eqnarray}

En définissant $c_n$ par
\begin{equation}
c_n = \frac{a_n - i b_n}{2}
\end{equation}
les deux énoncés suivants sont équivalents :
\begin{eqnarray}
\label{eqn:serfourreim}
x(t) & = & \sum_{n=-\infty}^{\infty} c_n \exp(in\omz t) \\
\label{eqn:coefffourreim}
c_n & = & \frac{1}{T}\intfou x(t)\exp(-in\omz t)dt
\end{eqnarray}
Ils constituent une reformulation de \ref{eqn:serfourdef} où la décomposition
d'une fonction périodique de période $T$
s'éffectue sur l'ensemble des fonctions exponentielles
complexes $\exp(in\omz t)$ ($-\infty < n < \infty, n \in \mathbb{Z}$).

Si la fonction périodique $x(t)$ est à valeurs complexes alors ses parties réelle
$x'(t)$ et imaginaire $x''(t)$ sont aussi périodiques et susceptibles
d'être décomposées en séries de Fourier :
\begin{eqnarray}
x(t) & = & x'(t) + i x''(t) \\
c_n(t) & = & c'_n(t) + i c''_n(t)
\end{eqnarray}
où les coefficients $c'_n$ et $c''_n$ sont respectivement 
issus de la décomposition de $x'(t)$ et $x''(t)$.
Les relations \ref{eqn:serfourreim} et \ref{eqn:coefffourreim}
s'appliquent donc aussi aux fonctions périodiques à valeurs complexes.

\section{Transformation de Fourier}
L'analyse d'une fonction par TF constitue une extension du concept de
décomposition en série de Fourier lorsque cette fonction n'est pas périodique.
On considère alors la fonction $\hat{x}(t)$ périodique de période $T$ telle que
\begin{equation}
\hat{x}(t) = x(t) \qpourq -\frac{T}{2} \le t \le \frac{T}{2}
\end{equation}

Lorsque $T$ tend vers l'infini, $\hat{x}(t)$ tend vers $x(t)$, $\omz$ tend vers 0,
et donc la différence de pulsations entre deux harmoniques consécutives
tend vers 0. Les définitions 
\begin{equation}
\omz = d\omega \quad \omega = n\omz \qetq \hat{y}(\omz) = c_n
\end{equation}
et l'approximation de la somme discrète du membre de gauche de l'équation
\ref{eqn:serfourreim} conduisent aux relations :
\begin{eqnarray}
\omz \hat{x}(t) & = & \intmp \hat{y}(\omega)\exp(i\omega t) d\omega \\
\hat{y}(\omega) & = & \frac{1}{T} \intfoub \hat{x}(t) \exp(-i\omega t) dt
\end{eqnarray}
Le passage à la limite quand $T$ tend vers l'infini permet d'écrire l'identité
\begin{equation}
x(t) = \frac{1}{2\pi}\intmp
\left[\intmp x(t) \exp(-i\omega t) dt
\right] \exp(i\omega t) d\omega
\end{equation}
sous réserve que les intégrales en question convergent.
Cette dernière relation conduit à définir la fonction $y(\omega)$
comme TF de $x(t)$ et $x(t)$ comme la TF inverse de $y(\omega)$ :
\begin{eqnarray}
\label{eqn:tfinv}
x(t) & = & K \intmp y(\omega) \exp(i\omega t) d\omega \\
\label{eqn:tf}
y(\omega) & = & K' \intmp x(t) \exp(-i\omega t) dt \\
2\pi K K' & = & 1
\end{eqnarray}

Plusieurs attitudes sont possibles vis-à-vis de $K$ et de $K'$ :
\begin{itemize}
\item le mépris : $K = K' = 1$,
\item l'attachement au concept de synthèse de Fourier où $x(t)$ est une somme de
fonctions exponentielles complexes : $K=1$ et $K' = 1/2\pi$,
\item l'attachement à la symétrie formelle entre $x(t)$ et $y(\omega)$ :
$K = K' = 1/\sqrt{2\pi}$.
\end{itemize}

\section{TF d'une fonction réelle}
Si $x(t)$ est une fonction à valeurs réelle, les parties réelles
et imaginaires de sa TF s'écrivent
\begin{eqnarray}
\label{eqn:tfrere}
\mbox{Re}[y(\omega)] & = & \intmp x(t) \cos(\omega t) dt \\
\label{eqn:tfreim}
\mbox{Im}[y(\omega)] & = & \intmp x(t) \sin(\omega t) dt 
\end{eqnarray}
Lorsque $x(t)$ représente un signal \emph{causal}, c'est-à-dire tel
que $x(t)=0$ si $t<0$, alors la borne de inférieure de l'intégrale
dans les équations \ref{eqn:tfrere} et \ref{eqn:tfreim} est remplacée par $0$
comme dans les équations \ref{eqn:ftftr} et \ref{eqn:ftfti}.
