\chapter{Référentiel tournant}
\label{chap:refer}

Le référentiel tournant, défini au paragraphe \ref{sec:rf}, tourne
autour de l'axe $Oz$ du référentiel du laboratoire à la fréquence $\omrf$,
fréquence des impulsions de radio-fréquence appliquées à l'échantillon
(figure \ref{fig:refer}).
Ce changement de repère est introduit pour faciliter la résolution de
l'équation \ref{eqn:blochrf1} :
\begin{equation}
\label{eqn:rf}
\frac{\mbox{d}\aimvec}{\mbox{d}t} =
\gamma \aimvec \wedge (\bzerovec + \bunvect)
\end{equation}

Les vecteurs de base $\iprimvec$, $\jprimvec$ et $\kprimvec$ du référentiel tournant
se déduisent des vecteurs de base $\ivec$, $\jvec$ et $\kvec$
du référentiel du laboratoire par les relations :
\begin{eqnarray}
\iprimvec & = & \cos(\omrft) \cdot \ivec + \sin(\omrft) \cdot \jvec \nonumber\\
\label{eqn:fix2mob}
\jprimvec & = & - \sin(\omrft) \cdot \ivec + \cos(\omrft) \cdot \jvec \\
\kprimvec & = & \kvec \nonumber
\end{eqnarray}
Les formules de transformation inverse s'écrivent :
\begin{eqnarray}
\ivec & = & \cos(\omrft) \cdot \iprimvec - \sin(\omrft) \cdot \jprimvec \nonumber\\
\label{eqn:mob2fix}
\jvec & = & - \sin(\omrft) \cdot \iprimvec + \cos(\omrft) \cdot \jprimvec \\
\kvec & = & \kprimvec \nonumber
\end{eqnarray}
Le signe de $\omrf$ est choisi de manière à ce que le référentiel tournant
évolue dans le même sens que la précession de Larmor.

Les coordonnées du vecteur $\aimvec$ sont définies dans le repère fixe 
et dans le repère tournant par :
\begin{eqnarray}
\aimvec & = & \aimxs \cdot \ivec + \aimys \cdot \jvec + \aimzs \cdot \kvec \\
\aimvec & = & \aimxs' \cdot \iprimvec + \aimys' \cdot \jprimvec + \aimzs' \cdot \kprimvec
\end{eqnarray}
La variation d'un vecteur au cours du temps dépend du repère où elle est considérée. 
Ainsi un vecteur fixe dans le référentiel tournant (solidaire des axes $OX$, $OY$ et $OZ$) 
varie dans le référentiel du laboratoire.
Le signe prime ($'$) sera utilisé par la suite pour désigner la dérivée
par rapport au temps d'une grandeur vectorielle, considérée dans le référentiel tournant.

Pour évaluer $\derivttxt{\aimvec}$ dans le référentiel
du laboratoire, membre de gauche de l'équation \ref{eqn:rf},
à partir des coordonnées de $\aimvec$
dans le référentiel tournant il faut tenir compte de la variation 
des vecteurs de base de ce dernier dans le référentiel du laboratoire.
Pour cela, il faut exprimer $\aimvec$ dans le référentiel tournant :
\begin{eqnarray}
\derivt{\aimvec} & = &
\derivt{\aimxs'} \cdot \iprimvec + 
\aimxs' \cdot \derivt{\iprimvec} \nonumber\\
&  & + \derivt{\aimys'} \cdot \jprimvec + 
\aimys' \cdot \derivt{\jprimvec} \nonumber\\
\label{eqn:derivvec} &  & + \derivt{\aimzs'} \cdot \kprimvec + 
\aimzs' \cdot \derivt{\kprimvec}
\end{eqnarray}

La dérivée de $\aimvec$ par rapport au temps, calculée dans le référentiel tournant, n'est autre que :
\begin{eqnarray}
\left(\derivt{\aimvec}\right)' & = &
\derivt{(\aimxs'\cdot\iprimvec+\aimys'\cdot\jprimvec+\aimzs'\cdot\kprimvec)}
\nonumber\\
\label{eqn:derivvec1}
& = & \derivt{\aimxs'} \cdot \iprimvec +
\derivt{\aimys'} \cdot \jprimvec +
\derivt{\aimys'} \cdot \jprimvec +
\derivt{\aimzs'} \cdot \kprimvec
\end{eqnarray}
et qui est bien différente de la dérivée de $\aimvec$ vue dans le référentiel
du laboratoire.
La différence entre les deux fait intervenir les dérivées des vecteurs
de base du référentiel tournant vues dans le référentiel du laboratoire.
Ces dérivées se calculent aisément à partir des équations \ref{eqn:fix2mob} :
\begin{eqnarray}
\derivt{\iprimvec} & = & -\omrf\sin(\omrft)\cdot\ivec +
\omrf\cos(\omrft)\cdot\jvec = +\omrf\cdot\jprimvec \nonumber\\
\derivt{\jprimvec} & = & -\omrf\cos(\omrft)\cdot\ivec -
\omrf\sin(\omrft)\cdot\jvec = -\omrf\cdot\iprimvec \\
\derivt{\kprimvec} & = & 0 \nonumber
\end{eqnarray}
Ainsi :
\begin{eqnarray}
\aimx' \cdot \derivt{\iprimvec} +
\aimy' \cdot \derivt{\jprimvec} +
\aimz' \cdot \derivt{\kprimvec} & = &
-\omrf\aimy'\cdot\iprimvec + \omrf\aimx'\cdot\jprimvec \nonumber\\
\label{eqn:derivvec2} & = & (\omrf\kvec) \wedge \aimvec
\end{eqnarray}
En reportant les résultats \ref{eqn:derivvec1} et \ref{eqn:derivvec2} dans
\ref{eqn:derivvec} on obtient
\begin{equation}
\derivt{\aimvec} = \left(\derivt{\aimvec}\right)'
+ (\omrf\kvec) \wedge \aimvec
\end{equation}

La partie du champ $\bunvect$ qui tourne dans le même sens
que le mouvement de précession de Larmor
(équation \ref{eqn:bunvect}) s'exprime dans le référentiel tournant
à l'aide des équations de transformation \ref{eqn:mob2fix} :
\begin{eqnarray}
\bunvect & = & \bunmax\cos(\omrft+\phi)(\cos(\omrft)\cdot\iprimvec-\sin(\omrft)\cdot\jprimvec)
\nonumber\\
& & + \bunmax\sin(\omrft+\phi)(\sin(\omrft)\cdot\iprimvec+\cos(\omrft)\cdot\jprimvec) \\
& = & \bunmax(\cos(\omrft+\phi)\cos(\omrft) + \sin(\omrft+\phi)\sin(\omrft))\cdot\iprimvec
\nonumber\\
& & + \bunmax(-\cos(\omrft+\phi)\sin(\omrft) + \sin(\omrft+\phi)\cos(\omrft))\cdot\jprimvec
\nonumber\\
& = & \bunmax(\cos\phi\cdot\iprimvec+\sin\phi\cdot\jprimvec)
\end{eqnarray}

Comme attendu, $\bunvect$ est immobile dans le référentiel tournant.
L'angle de phase $\phi$ de l'impulsion est l'angle constant que fait $\bunvec$ avec l'axe $OX$.
Le vecteur unitaire colinéaire avec $\bunvec$ dans le référentiel tournant sera noté 
$\uvec$ :
\begin{equation}
\uvec = \cos\phi \cdot \iprimvec + \sin\phi \cdot \jprimvec
\end{equation}
conduisant à
\begin{equation}
\bunvect = \bunmax\uvec
\end{equation}
où la dépendance de $\bunvec$ en fonction du temps réside dans le fait
que $\uvec$ tourne par rapport au référentiel du laboratoire.

En posant
\begin{equation}
\omega_0 = -\gamma\buns \qetq \omuns = -\gamma\bunmax
\end{equation}
l'équation \ref{eqn:rf} retranscrite dans le référentiel tournant devient :
\begin{equation}
\left(\derivt{\aimvec}\right)' + \omrf\cdot\kvec\wedge\aimvec = 
\left(\frac{\mbox{d}\aimvec}{\mbox{d}t}\right)' + \omrf\cdot\kvec\wedge\aimvec = 
- \aimvec\wedge (\omega_0\cdot\kvec + \omuns\cdot\uvec)
\end{equation}
soit encore 
\begin{equation}
\left(\derivt{\aimvec}\right)' =
-\aimvec\wedge((\omega_0-\omrf)\cdot\kvec + \omuns\cdot\uvec)
\end{equation}

Dans le référentiel tournant le vecteur $(\omega_0-\omrf)\cdot\kvec + \omuns\cdot\uvec$
est immobile. 
L'équation du mouvement de $\aimvec$ examinée dans le référentiel tournant se ramène donc
formellement à celle du mouvement de $\aimvec$ dans le référentiel du laboratoire
en présence du seul champ $\bzerovec$ (figure \ref{fig:effectif}).
Il suffit de considérer que l'aimantation de l'échantillon est soumise à un champ magnétique
appelé \emph{champ effectif} $\bveceff$ défini par 
\begin{equation}
(\omega_0-\omrf)\cdot\kvec + \omuns\cdot\uvec = -\gamma\cdot\bveceff = \omveceff 
\end{equation}
Ainsi :
\begin{equation}
\left(\derivt{\aimvec}\right)' = \aimvec\wedge\bveceff 
\end{equation}

Dans le référentiel tournant le mouvement de $\aimvec$ soumis simultanément aux champs 
$\bzerovec$ et $\bunvect$ est un mouvement de précession autour de l'axe de 
$\bveceff$ à la fréquence $\omeff/2\pi$.
Lorsque le champ $\bunvect$ est nul, $\omeff$ vaut $\omega_0-\omrf$.
Ceci était prévisible puisqu'alors $\aimvec$ tourne autour de l'axe 
$Oz$ (ou $OZ$) à la pulsation $\omega_0$ dans le référentiel du laboratoire
et que le référentiel mobile tourne dans le même sens
à la pulsation $\omrf$ autour de ce même référentiel.
